%%% -*- mode: LaTeX; TeX-master: "main.tex"; -*-

\ifx\master\undefined
\documentclass[a4paper]{easychair}
\usepackage{submission}
\begin{document}
{\let\master\relax%%% -*- mode: LaTeX; TeX-master: "main.tex"; -*-

\ifx\master\undefined
\documentclass[a4paper]{easychair}
\usepackage{submission}
\begin{document}
\fi
%%%% PLEASE DO NOT EDIT ABOVE THIS LINE

\title{A \tlaplus Proof System}

\titlerunning{A \tlaplus Proof System}

% \volumeinfo
% 	{P. Rudnicki, G. Sutcliffe} % editors
% 	{2}                         % number of editors
% 	{KEAPPA 2008}               % event
% 	{1}                         % volume
% 	{1}                         % issue
% 	{1}                         % starting page number


%  Alphabetically by surname
\author{
  Kaustuv Chaudhuri \\
  INRIA \\
  \and
  Damien Doligez \\
  INRIA \\
  \and
  Leslie Lamport \\
  Microsoft Research \\
  \and
  Stephan Merz \\
  INRIA \& Loria
}

\authorrunning{Chaudhuri, Doligez, Lamport, and Merz}

\maketitle

%%%% PLEASE DO NOT EDIT BELOW THIS LINE
\ifx\master\undefined
{\let\master\relax %%% -*- mode: LaTeX; TeX-master: "main.tex"; -*-

\ifx\master\undefined
\documentclass[a4paper]{easychair}
\usepackage{submission}
\begin{document}
{\let\master\relax %%% -*- mode: LaTeX; TeX-master: "main.tex"; -*-

\ifx\master\undefined
\documentclass[a4paper]{easychair}
\usepackage{submission}
\begin{document}
\fi
%%%% PLEASE DO NOT EDIT ABOVE THIS LINE

\title{A \tlaplus Proof System}

\titlerunning{A \tlaplus Proof System}

% \volumeinfo
% 	{P. Rudnicki, G. Sutcliffe} % editors
% 	{2}                         % number of editors
% 	{KEAPPA 2008}               % event
% 	{1}                         % volume
% 	{1}                         % issue
% 	{1}                         % starting page number


%  Alphabetically by surname
\author{
  Kaustuv Chaudhuri \\
  INRIA \\
  \and
  Damien Doligez \\
  INRIA \\
  \and
  Leslie Lamport \\
  Microsoft Research \\
  \and
  Stephan Merz \\
  INRIA \& Loria
}

\authorrunning{Chaudhuri, Doligez, Lamport, and Merz}

\maketitle

%%%% PLEASE DO NOT EDIT BELOW THIS LINE
\ifx\master\undefined
{\let\master\relax \input{rearmatter}}
\end{document}
\fi

% LocalWords:  tex Rudnicki Sutcliffe KEAPPA Kaustuv Chaudhuri INRIA Doligez
% LocalWords:  Merz Loria
}
\fi
%%%% PLEASE DO NOT EDIT ABOVE THIS LINE

\bibliographystyle{plain}
\bibliography{submission}

%%%% PLEASE DO NOT EDIT BELOW THIS LINE
\ifx\master\undefined
\end{document}
\fi

% LocalWords:  tex Paxos
}
\end{document}
\fi

% LocalWords:  tex Rudnicki Sutcliffe KEAPPA Kaustuv Chaudhuri INRIA Doligez
% LocalWords:  Merz Loria
}
\fi
%%%% PLEASE DO NOT EDIT ABOVE THIS LINE

\appendix

% \def\nil{\bullet}
\def\nil{%
  \begin{picture}(0,1)(0,-3)%
    \circle*{2}
  \end{picture}%
}

The appendices are provided as an aid to the Programme Committee and
the reviewers of the submission. They contain a level of detail that
may be inappropriate for a final version, and any part of it that is
found to be invaluable in understanding the body of the paper will be
incorporated into the body.


\section{Details of the \PM}
\label{apx}

\ednote{KC}{Unblobbify the definition paragraphs in this page.}

\llnote{On a quick reading, this looks OK.  However, I think it's longer
than it needs to be.  Just because it isn't included in the page count
doesn't mean we shouldn't be as concise as possible.  Let's see if we
can cut unnecessary material.\\[.3em]
Also, we need to be consistent
in whether we write \emph{Def.} or \emph{Defn.} or \emph{Definition}.
In the main body of the text, I have eliminated such abbreviations
(e.g., using \emph{Section} instead of \emph{Sec.}).  But they're
OK in the appendix.
}

We shall now give a somewhat formal declarative specification of the
\PM and prove the key theorem~\ref{thm:meaning}. First, a quick
summary of the abstract syntax of \tlatwo proofs, ignoring the
stylistic aspects of their concrete representation.

\sdef{lev}#1{_{\s{#1}}}

\begin{defn}[\tlatwo Proof Language] \label{defn:proof-language}
  %
  % All constructs in the \tlatwo proof language have an associated
  % \emph{level}, which is a natural number representing the nesting
  % depth of the proof; a construct "\chi" of level "n" is written as
  % "\chi lev n". 
  \tlatwo \emph{proofs}, \emph{non-leaf proofs}, \emph{proof steps}
  and \emph{begin-step} tokens have the following syntax, where "n"
  ranges over natural numbers, "l" over labels, "e" over expressions,
  "\Phi" over lists of expressions, "o" over operator definitions,
  "\P" over sets of operator names, "\vec \beta" over lists of binders
  (\ie, constructs of the form "x" and "x \in e" used to build
  quantified expressions), and "\alpha" over expressions or \ASSUME
  \ldots \PROVE forms.

  \begin{quote} \itshape
    \begin{tabbing}
      (Proofs) \hspace{4.5em}  \= "\pi" \LSP \= "::=" \ \= "\OBVIOUS OR \OMITTED OR \BY\ \Phi\ \DEFS\ \P OR \Pi" \\
      (Non-leaf proofs) \> "\Pi" \> "::=" \> "\sigma_n.\ \QED\ \PROOF\ \pi" \\
                   \>             \> "\ \ \ |"  \> "\sigma_n.\ \tau\quad \Pi" \\
      (Proof steps) \> "\tau" \> "::=" \> "\USE\ \Phi\ \DEFS\ \P OR \HIDE\ \Phi\ \DEFS\ \P OR \DEFINE\ o" \\
                    \>          \> "\ \ \ |" \> "\HAVE\ e OR \TAKE\ \vec \beta OR \WITNESS\ \Phi" \\
                    \>          \> "\ \ \ |" \> "\alpha\ \PROOF\ \pi OR \SUFFICES\ \alpha\ \PROOF\ \pi 
                                                    OR \PICK\ \vec \beta : e\ \PROOF\ \pi" \\
      (Begin-step tokens) \> "\sigma_n" \> "::=" \> "\s n OR \s n l"
    \end{tabbing}
  \end{quote}
  A proof that is not a non-leaf proof is called a \emph{leaf
    proof}. The level numbers that occur in the begin-step tokens of
  the proof of a step (the "\pi" in "\alpha\ \PROOF\ \pi", \etc.) must be strictly higher than those of the step
  itself.
\end{defn}

\subsection{The Meta-Language}
\label{apx:context}

The \PM uses proofs in the \tlatwo proof language
(Def.~\ref{defn:proof-language}) to manipulate constructs in the
meta-language of \tlatwo. Although this meta-language naturally has no
representation in \tlatwo itself, we define its syntax formally as
follows.

\begin{defn}[Meta-Language] \label{defn:syntax}
  %
  The \tlatwo meta-language consists of \emph{obligations},
  \emph{assumptions} and \emph{definables} with the following syntax,
  where "e" ranges over \tlatwo expressions, "x" and "o" over \tlatwo
  identifiers, and "\vec x" over lists of \tlatwo identifiers.
  %
  \begin{quote}
    \begin{tabbing}
      (Obligations) \SP \= "\phi" \SP \LSP \= "::=" \ \= "\obl{h_1, \dotsc, h_n ||- e}" \` ("n \ge 0") \\
      (Assumptions) \> "h" \> "::=" \> "\NEW x OR o \DEF \delta OR \phi OR \hide{o \DEF \delta} OR \hide{\phi}" \\
      (Definables) \> "\delta" \> "::=" \> "\phi OR \LAMBDA\ \vec x : e"
    \end{tabbing}
  \end{quote}
  %
  The expression after "||-" in an obligation is called its
  \emph{goal}. An assumption written inside square brackets
  "\hide{\ }" is said to be \emph{hidden}; otherwise it is
  \emph{usable}. An assumption "h" can be \emph{made usable}, written
  "\unhide h", by removing the brackets surrounding it (if any).
  %
  A list of assumptions is called a \emph{context}, with the empty
  context written as "\nil"; we let "\G", "\D" and "\W" range over
  contexts, with "\G, \D" standing for the context that is the
  concatenation of "\G" and "\D". The context "\unhide{\G}" is "\G"
  with all its hidden assumptions made usable.
  % 
  The obligation "\obl{\nil ||- e}" is written simply as "e". The
  assumptions "\NEW x", "o \DEF \delta" and "\hide{o \DEF \delta}"
  \emph{bind} the identifiers "x" and "o" respectively. We write "x
  \in \G" if "x" is bound in "\G" and "x \notin \G" if "x" is not
  bound in "\G". The context "\G, h" is considered syntactically
  well-formed iff "h" does not bind an identifier already bound in
  "\G".
\end{defn}

An obligation is a statement that its goal follows from the
assumptions in its context. \tlatwo already defines such a statement
using \ASSUME \ldots \PROVE, but the context in such statements have
no hidden assumptions or definitions. (To simplify the presentation
slightly, we give the semantics of a slightly enhanced proof language
where proof steps are allowed to mention obligations instead of just
\tlatwo \ASSUME \ldots \PROVE statements.) We define an embedding of
obligations into Isabelle/\tlaplus propositions, which we take as the
ultimate primitives of the \tlatwo meta-logic.

\begin{defn} \label{defn:isabelle-embedding}
  %
  The Isabelle/\tlaplus \emph{embedding} "\isa{-}" of obligations,
  contexts and definables is as follows:
  %
  \begin{align*}
    % \begin{aligned}
    \isa{"\G ||- e"} &\ =\  "\isa{\unhide{\G}}\, e" \\[1ex]
    % \end{aligned}
    % \qquad \qquad \qquad
    % \begin{aligned}
    \isa{\nil} &\ =\  \\
    \isa{"\G, \NEW x"} &\ =\  \isa{\G}\,\And x. \\
    \isa{"\G, o \DEF \delta"} &\ =\  \isa{\G}\,\And o.\,\bigl(o \equiv \isa{\delta}\bigr)\ "==>" \\
    \isa{"\G, \phi"} &\ =\  \isa{\G}\,\bigl(\isa{\phi}\bigr)\ "==>" \\[1ex]
    % \end{aligned}
    \isa{"\LAMBDA\ \vec x : e"} &\ =\ "\lambda \vec x.\ e" % \\[1ex]
  \end{align*}
\end{defn}

\noindent
For example, "\isa{\NEW P, \hide{\obl{\NEW x ||- P(x)}} ||- \forall x
  : P(x)} = \And P.\ \left(\And x.\ P(x)\right) ==> \forall x : P(x)".
Note that usable and hidden assumptions are treated identically for
the verifiability of an obligation.

The embedding of ordinary \tlatwo expressions is the identity because
Isabelle/\tlaplus contains \tlatwo expressions as part of its object
syntax. Thus, we do not have to trust the embedding of ordinary
\tlatwo expressions, just that of the obligation language.  In
practice, some aspects of \tlatwo expressions, such as the
indentation-sensitive conjunction and disjunction lists, are sent by
the \PM to Isabelle using an indentation-insensitive
encoding. Moreover, as already mentioned, Isabelle/\tlaplus accepts
only constant level expressions. While Isabelle/\tlaplus can
implicitly generalize over the free identifiers in a lemma, we shall
be explicit about binding and consider obligations verifiable only if
they are closed.

\begin{defn}[Verifiability] \label{defn:verifiable} \mbox{}
  %
  The obligation "\obl{\G ||- e}" is said to \emph{be verifiable} iff
  it is closed and "\isa{\G ||- e}" is a well-typed proposition of
  Isabelle/\tlaplus and is certified by Isabelle kernel to follow from
  the axioms of the Isabelle/\tlaplus logic.
\end{defn}

We state, omitting the trivial proofs, a number of useful facts about
obligations (which are all theorems in Isabelle/\tlaplus). The last one
(fact~\ref{thm:classic}) is true because \tlaplus is based on
classical logic.

\begin{fac}[Definition] \label{thm:definition}
  % 
  If "\obl{\G, \NEW o, \D ||- e}" is verifiable, then "\obl{\G, o \DEF
    \delta, \D ||- e}" is verifiable.
\end{fac}

\begin{fac}[Weakening] \label{thm:weaken}
  %
  If "\obl{\G, \D ||- e}" is verifiable, then "\obl{\G, h, \D ||- e}"
  is verifiable.
\end{fac}

\begin{fac}[Expansion] \label{thm:expand}
  %
  If "\obl{\G, o \DEF \delta, \D ||- e}" is verifiable, then "\obl{\G, o
    \DEF \delta, \D[o := \delta] ||- e[o := \delta]}" is verifiable.
\end{fac}

\begin{fac}[Strengthening] \label{thm:delete}
  %
  If "\obl{\G, \NEW o, \D ||- e}" or "\obl{\G, o \DEF \delta, \D ||-
    e}" is verifiable and "o"
  is not free in "\obl{\D ||- e}", then "\obl{\G, \D ||- e}" is verifiable.
\end{fac}

\begin{fac}[Cut] \label{thm:cut}
  %
  If "\obl{\G, \D ||- e}" is verifiable and "\obl{\G, (\D ||- e), \W ||- f}" is verifiable,
  then "\obl{\G, \W ||- f}" is verifiable.
\end{fac}

\begin{fac} \label{thm:classic}
  %
  If "\obl{\G, \lnot e, \D ||- e}" is verifiable, then "\obl{\G, \D ||- e}"
  is verifiable.
\end{fac}

The \USE/\HIDE \DEFS steps change the visibility of definitions in a
context (Defn.~\ref{defn:use/hide} below). Note that changing the
visibility of a definition does not affect the verifiability of an obligation
because the Isabelle embedding (Defn.~\ref{defn:isabelle-embedding})
makes all hidden definitions usable.

\pagebreak[2]

\begin{defn} \label{defn:use/hide}
  %
  If "\G" is a context and "\P" a set of operator names, then:
  \begin{ecom}
  \item \emph{"\G" with "\P" made usable}, written "\G \USING \P", is
    constructed from "\G" by replacing all assumptions of the form
    "\hide{o \DEF \delta}" in "\G" with "o \DEF \delta" for every "o
    \in \P".
  \item \emph{"\G" with "\P" made hidden}, written "\G \HIDING \P", is
    constructed from "\G" by replacing all assumptions of the form "o
    \DEF \delta" in "\G" with "\hide{o \DEF \delta}" for every "o \in
    \P".
  \end{ecom}
\end{defn}

\def\refl#1{\bigl\|\,#1\,\bigr\|}

A sequence of binders "\vec \beta" in the \tlatwo expressions "\forall
\vec \beta : e" or "\exists \vec \beta : e" can be reflected as
assumptions.

\begin{defn}[Binding Reflection] \label{defn:binding-reflection}
  %
  If "\vec \beta" is a list of binders with each element of the form
  "x" or "x \in e", then the \emph{reflection} of "\vec \beta" as
  assumptions, written "\refl{\vec \beta}", is given inductively as
  follows.
  \begin{align*}
    \refl{\nil} &= \nil &
    \refl{\vec \beta, x} &= \refl{\vec \beta}, \NEW x &
    \refl{\vec \beta, x \in e} &= \refl{\vec \beta}, \NEW x, x \in e
  \end{align*}
\end{defn}

\subsection{Interpreting Proofs}
\label{apx:proof-transformation}

Let us recall some definitions from section~\ref{sec:obligations}.

\begin{defn}[Claims and Transformations] \label{defn:check/trans-def}
  %
  A \emph{claim} is a judgement of the form "\pi : \obl{\G ||- e}"
  where "\pi" is a \tlatwo proof. A \emph{transformation} is a
  judgement of the form "\sigma.\,\tau : \obl{\G ||- e} --> \obl{\D
    ||- f}" where "\sigma" is a begin-step token and "\tau" a proof
  step.
\end{defn}

The \PM generates proof obligations from claims using two mutually
recursive procedures, \textit{checking} and
\textit{transformation}. Each procedure is specified below using the
formalism of a \textit{primitive derivation}.

\begin{defn} \label{defn:primitive}
  %
  A \emph{primitive derivation} is a derivation constructed using
  inferences
  % note to concerned citizens: I don't mean "inference rule" here. A
  % derivation is made up of a tree of inferences, with each inference
  % following a schematic inference rule.
  of the form
  \begin{gather*}
    \Ic{E}{\DD_1 & \dotsb & \DD_n} \tag*{"(n \ge 0)"}
  \end{gather*}
  where "E" is either a claim or a transformation, and "\DD_1, ...,
  \DD_n" are primitive derivations or obligations. An obligation at the
  leaf of a primitive derivation is called a \emph{proof obligation},
  and we say that a primitive derivation \emph{generates} the set of
  proof obligations at its leaves.
\end{defn}

\begin{defn}[Checking and Transformation Procedures] \label{defn:check/trans-proc} \mbox{}
  \begin{ecom}
  \item The \tlatwo proof "\pi" \emph{checks} the obligation "\obl{\G
      ||- e}" if there is a primitive derivation concluding "\pi :
    \obl{\G ||- e}" using the inference rules below and transformation
    rules.
    %
    \begin{gather*}
      % \I[\OBVIOUS]{\OBVIOUS : \obl{"\G ||- e"}}{\OBVIOUS : \obl{"\G ||- e"}}
      % \SP
      % \I[\OMITTED]{\OMITTED : \obl{"\G ||- e"}}{\OMITTED : \obl{"\G ||- e"}}
      % \\[1ex]
      \I[\BY]{"\BY\ \Phi\ \DEFS\ \P : (\G
        ||- e)"}
        {"\s{0}.\ \USE\ \Phi\ \DEFS\ \P" : "(\G ||- e) --> (\D ||- f)"
         &
         "\OBVIOUS : (\D ||- f)"}
      \\[1ex]
      \I[\QED]{"\sigma.\ \QED\ \PROOF\ \pi : (\G ||- e)"}
        {"\pi : (\G ||- e)"}
      \SP
      \I[non-\QED]{"\sigma.\,\tau\ \ \Pi : (\G ||- e)"}
        {"\sigma.\,\tau : (\G ||- e) --> (\D ||- f)"
         &
         "\Pi : (\D ||- f)"}
    \end{gather*}  
  \item 
    The \tlatwo proof step "\sigma.\,\tau" \emph{transforms} the
    obligation "\obl{\G ||- e}" into "\obl{\D ||- f}" if there is a
    primitive derivation concluding "\sigma.\,\tau : \obl{\G ||- e} -->
    \obl{\D ||- f}" using the inference rules below and checking rules.
    %
    \begin{gather*}
      \I[\USE\ \DEFS]{"\sigma.\ \USE\ \Phi\ \DEFS\ \P : \obl{\G ||- e} --> \obl{\D ||- f}"}
        {"\sigma.\ \USE\ \Phi : \obl{\G \USING \P ||- e} --> \obl{\D ||- f}"}
      \\[1ex]
      \I[\HIDE\ \DEFS]{"\sigma.\ \HIDE\ \Phi\ \DEFS\ \P : \obl{\G ||- e} --> \obl{\D \HIDING \P ||- f}"}
         {"\sigma.\ \HIDE\ \Phi : \obl{\G ||- e} --> \obl{\D ||- f}"}
      \\
      \I[\DEFINE ("o \notin \G")]{"\sigma.\ \DEFINE\ o \DEF \delta : \obl{\G ||- e} --> \obl{\G, \hide{o \DEF \delta} ||- e}"}
      \\
      \I["\USE_0"]{"\sigma.\ \USE\ \nil : \obl{\G ||- e} --> \obl{\G ||- e}"}
      \SP
      \I["\HIDE_0"]{"\sigma.\ \HIDE\ \nil : \obl{\G ||- e} --> \obl{\G ||- e}"}
      \\[1ex]
      \I["\USE_1"]{"\sigma.\ \USE\ \Phi, \obl{\G_0 ||- e_0} : \obl{\G ||- e} --> \obl{\D, \obl{\G_0 ||- e_0} ||- f}"}
        {"\sigma.\ \USE\ \Phi : \obl{\G ||- e} --> \obl{\D ||- f}"
         &
         "\OBVIOUS : \obl{\unhide{\D}, \G_0 ||- e_0}"
        }
      \\[1ex]
      \I["\HIDE_1"]{"\sigma.\ \HIDE\ \Phi, \phi : \obl{\G_0, \phi, \G_1 ||- e} --> \obl{\D ||- f}"}
        {"\sigma.\ \HIDE\ \Phi : \obl{\G_0, hide{\phi}, \G_1 ||- e} --> \obl{\D ||- f}"}
      \\
      \I["\TAKE_0"]{"\sigma.\ \TAKE\ \nil : \obl{\G ||- e} --> \obl{\G ||- e}"}
      \LSP
      \I["\WITNESS_0"]{"\sigma.\ \WITNESS\ \nil : \obl{\G ||- e} --> \obl{\G ||- e}"}
      \\[1ex]
      \I["\TAKE_1"]{"\sigma.\ \TAKE\ u, \vec\beta : \obl{\G ||- \forall x : e} --> \obl{\D ||- f}"}
        {"\sigma.\ \TAKE\ \vec\beta : \obl{\G, \NEW u ||- e [x := u]} --> \obl{\D ||- f}"}
      \\[1ex]
      \I["\TAKE_2"]{"\sigma.\ \TAKE\ u \in T, \vec\beta : \obl{\G ||- \forall x \in S : e} --> \obl{\D ||- f}"}
        {"\OBVIOUS : \obl{\G ||- S \subseteq T}"
         &
         "\sigma.\ \TAKE\ \vec\beta : \obl{\G, \NEW u, u \in T ||- e [x := u]} --> \obl{\D ||- f}"}
      \\[1ex]
      \I["\WITNESS_1"]{"\sigma.\ \WITNESS\ w, \W : \obl{\G ||- \exists x : e} --> \obl{\D ||- f}"}
        {"\sigma.\ \WITNESS\ \W : \obl{\G ||- e [x := w]} --> \obl{\D ||- f}"}
      \\[1ex]
      \I["\WITNESS_2"]{"\sigma.\ \WITNESS\ w \in T, \W : \obl{\G ||- \exists x \in S : e} --> \obl{\D ||- f}"}
        {\begin{array}[b]{c}
            "\OBVIOUS : \obl{\G ||- T \subseteq S}" \\
            "\OBVIOUS : \obl{\G ||- w \in T}"
         \end{array}
         &
         "\sigma.\ \WITNESS\ \W : \obl{\G, w \in T ||- e [x := w]} --> \obl{\D ||- f}"}
      \\[1ex]
      \I[\HAVE]{"\sigma.\ \HAVE\ g : \obl{\G ||- e => f} --> \obl{\G, g ||- f}"}
        {"\OBVIOUS : \obl{\G, e ||- g}"}
      \\[1ex]
      \I[$\ASSERT_1$]{"\s n.\ \obl{\D ||- f}\ \PROOF\ \pi : \obl{\G ||- e} --> \obl{\G, \obl{\D ||- f} ||- e}"}
        {"\pi : \obl{\G, \hide{\lnot e}, \D ||- f}"}
      \\[1ex]
      \I[$\ASSERT_2$]{"\s n l.\ \obl{\D ||- f}\ \PROOF\ \pi : \obl{\G ||- e} --> \obl{\G, \s n l \DEF \obl{\D ||- f}, \hide{\s n l} ||- e}"}
        {"\pi : \obl{\G, \s n l \DEF \obl{\D ||- f}, \hide{\lnot e}, \D ||- f}"}
      \\[1ex]
      \I[\CASE]{"\sigma.\ \CASE\ g\ \PROOF\ \pi : \obl{\G ||- e}
                       --> \obl{\D ||- f}"}
        {"\sigma.\ \obl{g ||- e}\ \PROOF\ \pi : \obl{\G ||- e} --> \obl{\D ||- f}"}
      \\[1ex]
      \I[$\SUFFICES_1$]{"\s n.\ \SUFFICES\ \obl{\D ||- f}\ \PROOF\ \pi : \obl{\G ||- e} --> \obl{\G, \hide{\lnot e}, \D ||- f}"}
        {"\pi : \obl{\G, \obl{\D ||- f} ||- e}"}
      \\[1ex]
      \I[$\SUFFICES_2$]{"\s n l.\ \SUFFICES\ \obl{\D ||- f}\ \PROOF\ \pi : \obl{\G ||- e} --> \obl{\G, \s n l \DEF \obl{\D ||- f}, \hide{\lnot e}, \D ||- f}"}
        {"\pi : \obl{\G, \s n l \DEF \obl{\D ||- f}, \hide{\s n l} ||- e}"}
      \\[1ex]
      \I[\PICK]{"\sigma.\ \PICK\ \vec\beta : p\ \PROOF\ \pi :
                  \obl{\G ||- e} --> \obl{\G, \refl{\vec \beta}, p ||- e}"}
        {"\pi : \obl{\G ||- \exists \vec \beta : p}"}
    \end{gather*}
  \end{ecom}
\end{defn}

The inference rules in the two definitions above are deterministic:
the conclusion of each rule uniquely determines the premises. However,
the rules are partial; for example, there is no rule that concludes a
transformation of the form "\sigma.\,\TAKE\ x \in S : \obl{\G ||- B \land C}
--> \obl{\D ||- f}".

\begin{defn} \label{defn:meaningful}
  % 
  A claim or a transformation is said to be \emph{meaningful} if it
  has a primitive derivation.
\end{defn}

In the rest of this appendix we limit our attention to meaningful
claims and transformations.

\subsection{Correctness}
\label{apx:correctness}

If the proof obligations generated by a meaningful claim are
verifiable, then the obligation in the claim itself ought to be
verifiable. In this section we shall prove this theorem by analysis of
the checking and transformation derivations. The proof of this theorem
will moreover demonstrate how to produce a proof of the ``structural
theorem'' of a claim, which is a (meta-)implication from the
obligations of the proof obligations of the claim to the obligation of
the claim.

\begin{defn}[Verifiability of Claims and Transformation]
  %
  \label{defn:proc-truth} \mbox{}
  %
  \begin{ecom}
  \item The claim "\pi : \obl{\G ||- e}" is \emph{verifiable} iff it is
    meaningful and the proof obligations it generates are all verifiable.
  \item The transformation "\sigma.\,\tau : \obl{\G ||- e} --> \obl{\D
      ||- f}" is \emph{verifiable} iff it is meaningful and the proof
    obligations it generates are all verifiable.
  \end{ecom}
\end{defn}

\begin{thm}[Correctness] \label{thm:correctness} \mbox{}
  %
  \begin{ecom}[\LSP (1)]
  \item If "\pi : \obl{\G ||- e}" is verifiable, then "\obl{\G ||- e}"
    is verifiable.
  \item If "\sigma.\,\tau : \obl{\G ||- e} --> \obl{\D ||- f}" is
    verifiable and "\obl{\D ||- f}" is verifiable, then "\obl{\G ||-
      e}" is verifiable.
  \end{ecom}
\end{thm}

\begin{proof}
  Let "\DD" be the primitive derivation for the claim in (1) and "\EE"
  be the primitive derivation for the transformation in (2). The proof
  will be by lexicographic induction on the structures of "\DD" and
  "\EE", with a verifiable transformation allowed to justify a
  verifiable claim.

  \begin{ecom}[{$\s1$}1.]
  \item If "\pi : \obl{\G ||- e}" is verifiable, then "\obl{\G ||-
      e}" is verifiable.
    %
    \begin{ecom}[{$\s2$}1.] \setlength{\itemsep}{6pt}
    \item \Case "\pi" is "\BY\ \Phi\ \DEFS\ \P", \ie,
      \begin{gather*}
        \DD = 
        \Im[\BY.]{"\BY\ \Phi\ \DEFS\ \P : (\G ||- e)"}
           {\deduce{"\s{0}.\ \USE\ \Phi\ \DEFS\ \P" : "(\G ||- e) --> (\D ||- f)"}{\EE_0}
            &
            "\OBVIOUS : (\D ||- f)"}
      \end{gather*}
      \begin{ecom}[{$\s3$}1.]
      \item "\obl{\D ||- f}" is verifiable
        %
        \by Def.~\ref{defn:proc-truth}.
      \item \Qed
        %
        \by \s31, i.h. (inductive hypothesis) for "\EE_0".
      % \item \Qed
      %   %
      %   \by \s31, \s32.
      \end{ecom}

    \item \Case "\pi" is "\sigma.\, \QED\ \PROOF\ \pi_0", \ie, 
      $
      \DD =
      \Im[\QED.]{"\sigma.\ \QED\ \PROOF\ \pi_0 : (\G ||- e)"}
         {\deduce{"\pi_0 : (\G ||- e)"}{\DD_0}}$
      %
      \by i.h. for "\DD_0".

    \item \Case "\pi" is "\sigma.\,\tau\ \ \Pi", \ie,
      \begin{gather*}
        \DD =
        \Im[non-\QED.]{"\sigma.\,\tau\ \ \Pi : (\G ||- e)"}
          {\deduce{"\sigma.\,\tau : (\G ||- e) --> (\D ||- f)"}{\EE_0}
           &
           \deduce{"\Pi : (\D ||- f)"}{\DD_0}
          }
      \end{gather*}

      \begin{ecom}[{$\s3$}1.]
      \item [\s31.] "\obl{\D ||- f}" is verifiable
        % 
        \by i.h. for "\DD_0".
      \item [\s33.] \Qed
        %
        \by \s31, i.h. for "\EE_0".
      \end{ecom}
    \item \Qed \by \s21, \s22, \s23.
    \end{ecom}

  \item If "\sigma.\,\tau : \obl{\G ||- e} --> \obl{\D ||- f}" is
    verifiable and "\obl{\D ||- f}" is verifiable, then "\obl{\G ||-
      e}" is verifiable.
    %
    \begin{ecom}[{$\s2$}1.] \setlength{\itemsep}{6pt}

    \item \Case "\tau" is "\USE\ \Phi\ \DEFS\ \P", \ie,
      \begin{gather*}
        \EE =
        \Im[\USE\ \DEFS.]{"\sigma.\ \USE\ \Phi\ \DEFS\ \P : \obl{\G ||- e} --> \obl{\D ||- f}"}
          {\deduce{"\sigma.\ \USE\ \Phi : \obl{\G \USING \P ||- e} --> \obl{\D ||- f}"}{\EE_0}}
      \end{gather*}

      \begin{ecom}[{$\s3$}1.]
      \item "\obl{\G \USING \P ||- e}" is verifiable
        %
        \by i.h. for "\EE_0".
      \item \Qed
        %
        \by \s31, Defn.~\ref{defn:use/hide}.
      \end{ecom}

    \item \Case "\tau" is "\HIDE\ \Phi\ \DEFS\ \P", \ie,
      \begin{gather*}
        \EE =
        \Im[\HIDE\ \DEFS.]{"\sigma.\ \HIDE\ \Phi\ \DEFS\ \P : \obl{\G ||- e} --> \obl{\D \HIDING \P ||- f}"}
           {\deduce{"\sigma.\ \HIDE\ \Phi : \obl{\G ||- e} --> \obl{\D ||- f}"}{\EE_0}}
      \end{gather*}

      \begin{ecom}[{$\s3$}1.]
      \item "\obl{\D ||- f}" is verifiable
        %
        \by verifiability of "\obl{\D \HIDING \P ||- f}" and Defn.~\ref{defn:use/hide}.
      \item \Qed
        %
        \by \s31, i.h. for "\EE_0".
      \end{ecom}

    \item \Case "\tau" is "\DEFINE\ o \DEF \delta" with "o \notin \G", \ie, 
      \begin{gather*}
        \EE =
        \Im[\DEFINE.]{"\sigma.\ \DEFINE\ o \DEF \delta : \obl{\G ||- e} --> \obl{\G, \hide{o \DEF \delta} ||- e}"}{}
      \end{gather*}

      \begin{ecom}[{$\s3$}1.]
      \item "o" is not free in "e"
        %
        \by "o \notin \G" and closedness of "\obl{\G ||- e}".
      \item \Qed
        %
        \by \s31, strengthening (fact~\ref{thm:delete}).
      \end{ecom}

    \item \Case "\tau" is "\USE\ \nil", \ie,
      $
      \EE = 
      \Im["\USE_0".]{"\sigma.\ \USE\ \nil : \obl{\G ||- e} --> \obl{\G ||- e}"}{}
      $
      \Trivial

    \item \Case "\tau" is "\HIDE\ \nil", \ie,
      $
      \EE = 
      \Im["\HIDE_0".]{"\sigma.\ \HIDE\ \nil : \obl{\G ||- e} --> \obl{\G ||- e}"}{}
      $
      \Trivial

    \item \Case "\tau" is "\USE\ \Phi, \phi", \ie,
      \begin{gather*}
        \EE = 
        \Im["\USE_1"]{"\sigma.\ \USE\ \Phi, \obl{\G_0 ||- e_0} : \obl{\G ||- e} --> \obl{\D_0, \obl{\G_0 ||- e_0} ||- f}"}
          {\deduce{"\sigma.\ \USE\ \Phi : \obl{\G ||- e} --> \obl{\D_0 ||- f}"}{\EE_0}
           &
           "\OBVIOUS : \obl{\unhide{\D_0}, \G_0 ||- e_0}"
          }      
      \end{gather*}

      \begin{ecom}[{$\s3$}1.]
      \item "\obl{\unhide{\D_0}, \G_0 ||- e_0}" is verifiable
        %
        \by Def.~\ref{defn:proc-truth}.
      \item "\obl{\D_0, \G_0 ||- e_0}" is verifiable
        %
        \by \s31, Def.~\ref{defn:isabelle-embedding}.
      \item "\obl{\D_0 ||- f}" is verifiable
        %
        \by verifiability of "\obl{\D_0, (\G_0 ||- e_0) ||- f}", \s32, cut (fact~\ref{thm:cut}).
      \item \Qed
        %
        \by \s33, i.h. for "\EE_0"
      \end{ecom}

    \item \Case "\tau" is "\HIDE\ \Phi, \phi", \ie,
      \begin{gather*}
        \EE =
        \Im["\HIDE_1".]{"\sigma.\ \HIDE\ \Phi, \phi : \obl{\G_0, \phi, \G_1 ||- e} --> \obl{\D ||- f}"}
           {\deduce{"\sigma.\ \HIDE\ \Phi : \obl{\G_0, hide{\phi}, \G_1 ||- e} --> \obl{\D ||- f}"}{\EE_0}}
      \end{gather*}

      \begin{ecom}[{$\s3$}1.]
      \item "\obl{\G_0, \hide{\phi}, \G_1 ||- e}" is verifiable
        %
        \by verifiability of "\obl{\D ||- f}", i.h. for "\EE_0".
      \item \Qed
        %
        \by \s31, "\isa{\G_0, \hide{\phi}, \G_1 ||- e} = \isa{\G_0, \phi, \G_1 ||- e}" (Def.~\ref{defn:isabelle-embedding}).
      \end{ecom}

    \item \Case "\tau" is "\TAKE\ \nil", \ie,
      $
      \EE = 
      \Im["\TAKE_0".]{"\sigma.\ \TAKE\ \nil : \obl{\G ||- e} --> \obl{\G ||- e}"}{}
      $
      \Trivial

    \item \Case "\tau" is "\WITNESS\ \nil", \ie,
      $
      \EE = 
      \Im["\WITNESS_0".]{"\sigma.\ \WITNESS\ \nil : \obl{\G ||- e} --> \obl{\G ||- e}"}{}
      $
      \Trivial

    \item \Case "\tau" is "\TAKE\ u, \vec \beta", \ie,
      \begin{gather*}
        \EE =
        \Im["\TAKE_1".]{"\sigma.\ \TAKE\ u, \vec\beta : \obl{\G ||- \forall x : e} --> \obl{\D ||- f}"}
           {\deduce{"\sigma.\ \TAKE\ \vec\beta : \obl{\G, \NEW u ||- e [x := u]} --> \obl{\D ||- f}"}{\EE_0}}
      \end{gather*}

      \begin{ecom}[{$\s3$}1.]
      \item "\obl{\G, \NEW u ||- e [x := u]}" is verifiable
        %
        \by i.h. for "\EE_0".
      \item \Qed
        %
        \by \s31{} and predicate logic.
% original version
%        \by \s32.
      \end{ecom}
% \ednote{SM}{I changed the justification of the QED step, which didn't
%   make sense. Kaustuv, please check.}
%
% OK, I agree. -- KC

    \item \Case "\tau" is "\sigma.\ \TAKE\ u \in T", \ie,
      \begin{gather*}
        \EE =
        \Im["\TAKE_2".]{"\sigma.\ \TAKE\ u \in T, \vec\beta : \obl{\G ||- \forall x \in S : e} --> \obl{\D ||- f}"}
           {"\OBVIOUS : \obl{\G ||- S \subseteq T}"
            &
            \deduce{"\sigma.\ \TAKE\ \vec\beta : \obl{\G, \NEW u, u \in T ||- e [x := u]} --> \obl{\D ||- f}"}{\EE_0}}
      \end{gather*}

      \begin{ecom}[{$\s3$}1.]
      \item "\obl{\G, \NEW u, u \in T ||- e [x := u]}" is verifiable
        %
        \by i.h on "\EE_0".
      \item "\obl{\G, \NEW u, u \in S ||- u \in T}" is verifiable
        \begin{ecom}[{$\s4$}1.]
        \item "\obl{\G, \NEW u ||- S \subseteq T}" is verifiable
          %
          \by Def.~\ref{defn:proc-truth}, weakening
          (fact~\ref{thm:weaken}).
        \item \Qed
          %
          \by \s41, Def. of "\subseteq".
        \end{ecom}
      \item "\obl{\G, \NEW u, u \in S ||- e[x := u]}" is verifiable
        %
        \by \s31, \s32, cut (fact~\ref{thm:cut}).
      \item \Qed
        %
        \by \s33{} and predicate logic.
      \end{ecom}
% \ednote{SM}{I added ``and predicate logic'': if you edit, please keep
%   it consistent with the preceding step.}
%
% No need to edit it. I agree with your modification. -- KC

    \item \Case "\tau" is "\WITNESS\ w, \W", \ie,
      \begin{gather*}
        \EE =
        \Im["\WITNESS_1".]{"\sigma.\ \WITNESS\ w, \W : \obl{\G ||- \exists x : e} --> \obl{\D ||- f}"}
           {\deduce{"\sigma.\ \WITNESS\ \W : \obl{\G ||- e [x := w]} --> \obl{\D ||- f}"}{\EE_0}}
      \end{gather*}

      \begin{ecom}[{$\s3$}1.]
      \item "\obl{\G ||- e [x := w]}" is verifiable
        %
        \by i.h. for "\EE_0".
      \item \Qed
        %
        \by \s31.
      \end{ecom}

    \item \Case "\tau" is "\WITNESS\ w \in T, \W" and:
      \begin{gather*}
        \EE =
        \Im["\WITNESS_2".]
           {"\sigma.\ \WITNESS\ w \in T, \W : \obl{\G ||- \exists x \in S : e} --> \obl{\D ||- f}"}
           {\begin{array}[b]{c}
               "\OBVIOUS : \obl{\G ||- T \subseteq S}" \\
               "\OBVIOUS : \obl{\G ||- w \in T}"
            \end{array}
            &
            \deduce{"\sigma.\ \WITNESS\ \W : \obl{\G, w \in T ||- e [x := w]} --> \obl{\D ||- f}"}{\EE_0}}
     \end{gather*}

     \begin{ecom}[{$\s3$}1.]
     \item "\obl{\G, w \in T ||- e [x := w]}" is verifiable
       % 
       \by i.h. for "\EE_0".
     \item "\obl{\G ||- w \in T}" is verifiable
       % 
       \by Def.~\ref{defn:proc-truth}.
     \item "\obl{\G ||- e [x := w]}" is verifiable
       % 
       \by \s31, \s32, cut (fact~\ref{thm:cut}).
     \item "\obl{\G ||- w \in S}" is verifiable
       \begin{ecom}[{$\s4$}1.]
       \item "\obl{\G, w \in T ||- w \in S}" is verifiable
         % 
         \by Def.~\ref{defn:proc-truth}, Def. of "\subseteq".
       \item \Qed
         % 
         \by \s41, \s32, cut (fact~\ref{thm:cut}).
       \end{ecom}
     \item \Qed
       % 
       \by \s33, \s34, and predicate logic.
     \end{ecom}
     
    \item "\tau" is "\HAVE\ g", \ie,
      \begin{gather*}
        \EE =
        \Im[\HAVE.]{"\sigma.\ \HAVE\ g : \obl{\G ||- e => f} --> \obl{\G, g ||- f}"}
           {"\OBVIOUS : \obl{\G, e ||- g}"}
      \end{gather*}

      \begin{ecom}[{$\s3$}1.]
      \item "\obl{\G, e, g ||- f}" is verifiable
        %
        \by weakening (fact~\ref{thm:weaken}).
      \item "\obl{\G, e ||- g}" is verifiable
        %
        \by Def.~\ref{defn:proc-truth}.
      \item "\obl{\G, e ||- f}" is verifiable
        %
        \by \s31, \s32, cut (fact~\ref{thm:cut}).
      \item "\obl{\G ||- e => f}" is verifiable
        %
        \by \s33.
      \end{ecom}

    \item "\sigma.\,\tau" is "\s n.\ \obl{\W ||- g}\ \PROOF\ \pi", \ie, 
      \begin{gather*}
        \EE =
        \Im[$\ASSERT_1$.]{"\s n.\ \obl{\W ||- g}\ \PROOF\ \pi : \obl{\G ||- e} --> \obl{\G, \obl{\W ||- g} ||- e}"}
           {\deduce{"\pi : \obl{\G, \hide{\lnot e}, \W ||- g}"}{\DD_0}}
      \end{gather*}

      \begin{ecom}[{$\s3$}1.]
      \item "\obl{\G, \hide{\lnot e}, \obl{\W ||- g} ||- e}" is verifiable
        %
        \by weakening (fact~\ref{thm:weaken}).
      \item "\obl{\G, \hide{\lnot e}, \W ||- g}" is verifiable
        %
        \by i.h. for "\DD_0".
      \item "\obl{\G, \hide{\lnot e} ||- e}" is verifiable
        %
        \by \s31, \s32, cut (fact~\ref{thm:cut}).
      \item \Qed
        %
        \by \s33, fact~\ref{thm:classic}.
      \end{ecom}

    \item \Case "\sigma.\,\tau" is "\s n l.\ \obl{\W ||- g}\ \PROOF\ \pi", \ie,
      \begin{gather*}
        \EE =
        \Im[$\ASSERT_2$.]{"\s n l.\ \obl{\W ||- g}\ \PROOF\ \pi : \obl{\G ||- e} --> \obl{\G, \s n l \DEF \obl{\W ||- g}, \hide{\s n l} ||- e}"}
           {\deduce{"\pi : \obl{\G, \s n l \DEF \obl{\W ||- g}, \hide{\lnot e}, \W ||- g}"}{\DD_0}}
      \end{gather*}

      \begin{ecom}[{$\s3$}1.]
      \item "\obl{\G, \s n l \DEF \obl{\W ||- g}, \hide{\lnot e}, \hide{\s n l} ||- e}" is verifiable
        \\\mbox{}
        \by verifiability of "\obl{\G, \s n l \DEF \obl{\W ||- g}, \hide{\s n l} ||- e}", weakening (fact~\ref{thm:weaken}).
      \item "\obl{\G, \s n l \DEF \obl{\W ||- g}, \hide{\lnot e}, \hide{\obl{\W ||- g}} ||- e}" is verifiable
        %
        \by \s31, expansion (fact~\ref{thm:expand}).
      \item "\obl{\G, \s n l \DEF \obl{\W ||- g}, \hide{\lnot e}, \W ||- g}" is verifiable
        % 
        \by i.h. for "\DD_0", weakening.
      \item "\obl{\G, \s n l \DEF \obl{\W ||- g}, \hide{\lnot e} ||- e}" is verifiable
        %
        \by \s32, \s33, cut (fact~\ref{thm:cut}).
      \item "\obl{\G, \hide{\lnot e} ||- e}" is verifiable
        %
        \by \s34, strengthening (fact~\ref{thm:delete}).
      \item \Qed
        %
        \by \s35, fact~\ref{thm:classic}.
      \end{ecom}
      
    \item "\tau" is "\CASE\ g\ \PROOF\ \pi", \ie,
      \begin{gather*}
        \EE =
        \Im[\CASE.]{"\sigma.\ \CASE\ g\ \PROOF\ \pi : \obl{\G ||- e} --> \obl{\D ||- f}"}
           {\deduce{"\sigma.\ \obl{g ||- e}\ \PROOF\ \pi : \obl{\G ||- e} --> \obl{\D ||- f}"}{\EE_0}}
      \end{gather*}

      \textit{By} i.h. for "\EE_0".

    \item "\tau" is "\s n.\ \SUFFICES\ \obl{\W ||- g}\ \PROOF\ \pi", \ie,
      \begin{gather*}
        \EE =
        \Im[$\SUFFICES_1$.]{"\s n.\ \SUFFICES\ \obl{\D ||- f}\ \PROOF\ \pi : \obl{\G ||- e} --> \obl{\G, \hide{\lnot e}, \W ||- g}"}
           {\deduce{"\pi : \obl{\G, \obl{\W ||- g} ||- e}"}{\DD_0}}
      \end{gather*}

    \begin{ecom}[{$\s3$}1.]
      \item "\obl{\G, \hide{\lnot e}, \obl{\W ||- g} ||- e}" is verifiable
        %
        \by i.h. for "\DD_0", weakening (fact~\ref{thm:weaken}).
      \item "\obl{\G, \hide{\lnot e} ||- e}" is verifiable
        %
        \by verifiability of "\obl{\G, \hide{\lnot e}, \W ||- g}", \s31, cut (fact~\ref{thm:cut}).
      \item \Qed
        %
        \by \s32, fact~\ref{thm:classic}.
      \end{ecom}

    \item "\sigma.\,\tau" is "\s n l.\ \SUFFICES\ \obl{\W ||- g}\ \PROOF\ \pi", \ie,
      \begin{gather*}
        \EE =
        \Im[$\SUFFICES_2$.]{"\s n l.\ \SUFFICES\ \obl{\W ||- g}\ \PROOF\ \pi : \obl{\G ||- e} --> \obl{\G, \s n l \DEF \obl{\W ||- g}, \hide{\lnot e}, \W ||- g}"}
           {\deduce{"\pi : \obl{\G, \s n l \DEF \obl{\W ||- g}, \hide{\s n l} ||- e}"}{\DD_0}}
      \end{gather*}

      \begin{ecom}[{$\s3$}1.]
      \item "\obl{\G, \s n l \DEF \obl{\W ||- g}, \hide{\lnot e}, \hide{\s n l} ||- e}" is verifiable
        %
        \by i.h. for "\DD_0", weakening (fact~\ref{thm:weaken}).
      \item "\obl{\G, \s n l \DEF \obl{\W ||- g}, \hide{\lnot e}, \hide{\obl{\W ||- g}} ||- e}" is verifiable
        %
        \by \s31, expansion (fact~\ref{thm:expand}).
      \item "\obl{\G, \s n l \DEF \obl{\W ||- g}, \hide{\lnot e} ||- e}" is verifiable
        \\\mbox{}
        \by \s32, verifiability of "\obl{\G, \s n l \DEF \obl{\W ||- g}, \hide{\lnot e}, \W ||- g}", cut (fact~\ref{thm:cut}).
      \item "\obl{\G, \hide{\lnot e} ||- e}" is verifiable
        %
        \by \s33, strengthening (fact~\ref{thm:delete}).
      \item \Qed
        %
        \by \s34, fact~\ref{thm:classic}.
      \end{ecom}

    \item \Case "\tau" is "\PICK\ \vec\beta : p\ \PROOF\ \pi", \ie,
      \begin{gather*}
        \EE =
        \Im[\PICK.]{"\sigma.\ \PICK\ \vec\beta : p\ \PROOF\ \pi :
                        \obl{\G ||- e} --> \obl{\G, \refl{\vec \beta}, p ||- e}"}
           {\deduce{"\pi : \obl{\G ||- \exists \vec \beta : p}"}{\DD_0}}
      \end{gather*}

      \begin{ecom}[{$\s3$}1.]
      \item "\obl{\G, \exists \vec \beta : p ||- e}" is verifiable
        %
        \by verifiability of "\obl{\G, \refl{\vec \beta}, p ||- e}", predicate logic.
      \item "\obl{\G ||- \exists \vec \beta : p}" is verifiable
        %
        \by i.h. for "\DD_0".
      \item \Qed
        %
        \by \s31, \s32, cut (fact~\ref{thm:cut}).
      \end{ecom}

    \item \Qed
      %
      \by \s21, \dots, \s220
    \end{ecom}
  \item \Qed
    %
    \by \s11, \s12.
  \end{ecom}
\end{proof}

\subsection{Constrained Search}
\label{apx:constraints}

The correctness theorem (\ref{thm:correctness}) establishes an
implication from the proof obligations generated by a meaningful claim
to its obligation. It is always true, regardless of the verifiability
of any individual obligation. While changing the visibility of
assumptions in an obligation does not change its verifiability, it can
change its verifiability if important assumptions are hidden.  As
already mentioned in Section~\ref{sec:obligations}, the \PM removes
these hidden assumptions before sending a proof obligation to a
back-end prover.  Therefore, in order to establish the structural
soundness theorem (\ref{thm:meaning}), we must prove a property about
the result of this removal.

\begin{defn}[Filtration] \label{defn:filter} 
  %
  The \emph{filtered} form of any obligation "\phi", written
  "\filter\phi", is obtained by deleting all assumptions of the form
  "\hide{\phi_0}" and replacing all assumptions of the form "\hide{o
    \DEF \delta}" with "\NEW o" anywhere inside "\phi".
\end{defn}

For example, if "\phi" is "\obl{\NEW x, \hide{y \DEF x} ||- x = y}",
then "\filter\phi = \obl{\NEW x, \NEW y ||- x = y}". Clearly,
filtration makes an obligation implicationally weaker, for "\phi" is
verifiable while "\filter\phi" is not.

\begin{lem}[Verification Lemma] \label{thm:verification}
  %
  If "\filter{\phi}" is verifiable, then "\phi" is verifiable.
\end{lem}

\begin{proof}[Proof Sketch]
  By induction on the structure of the obligation "\phi", with each
  case a straightforward consequence of facts~\ref{thm:definition}
  and~\ref{thm:weaken}.
\end{proof}

We now prove the structural soundness theorem (\ref{thm:meaning}).

\setcounter{thm}{0}

\begin{thm}
  % 
  If "\pi:\phi" is a meaningful claim, and for every proof obligations
  "\phi" it generates, "\filter{\phi}" is verifiable, then "\phi" is
  verifiable.
\end{thm}

\begin{proof}\mbox {}
  \begin{ecom}[{$\s1$}1.]
  \item For every proof obligation "\phi_0" generated by "\pi : \phi",
    it must be that "\phi_0" is verifiable.
    \begin{ecom}[{$\s2$}1.]
    \item \textit{Take} "\phi_0" as a proof obligation generated by
      "\pi : \phi".
    \item "\filter{\phi_0}" is verifiable \by assumptions.
    \item \Qed \by \s22, verification lemma~\ref{thm:verification}.
    \end{ecom}
  \item \Qed \by correctness theorem~\ref{thm:correctness}.
  \end{ecom}
\end{proof}

\clearpage

\section{A \tlatwo Proof of Cantor's Theorem}
\label{apx:cantor}

The following is the full \tlatwo proof of Cantor's theorem.
%
\begin{quote} \small
  \begin{tabbing}
    \THEOREM\ "\forall S : \forall f \in [S -> \SUBSET\ S] : \exists A \in \SUBSET\ S : \forall x \in S : f[x] \neq A" \\
    \s11.\ \= \ASSUME \= "\NEW\ S", \\
           \>         \> "\NEW\ f \in [S -> \SUBSET\ S]" \\
           \> \PROVE "\exists A \in \SUBSET\ S : \forall x \in S : f[x] \neq A" \\
           \> \s21.\ \= \DEFINE "T \DEF \{z \in S : z \notin f[z]\}" \\
           \> \s22.  \> "\forall x \in S : f[x] \neq T" \\
           \>        \> \s31.\ \= \ASSUME "\NEW\ x \in S" \PROVE "f[x] \neq T" \\
           \>        \>        \> \s41.\ \= \CASE "x \in T" \OBVIOUS \\
           \>        \>        \> \s42.\ \> \CASE "x \notin T" \OBVIOUS \\
           \>        \>        \> \s43.\ \> \QED \BY \s41, \s42 \\
           \>        \> \s32.  \> \QED\ \BY\ \s31 \\
           \> \s23.  \> \QED\ \BY\ \s22 \\
    \s12.  \> \QED\ \BY\ \s11
  \end{tabbing}
\end{quote}
%
As an example, the obligation generated (see Appendix~\ref{apx:correctness}) for the proof of \s41 is:
%
\begin{quote} \small
  \begin{tabbing}
    "\OBVIOUS : \smash{\Bigl(}"
    \= "\s11 \DEF \obl{\NEW S, \NEW f, f \in [S -> \SUBSET\ S] ||- \exists A \in \SUBSET\ S : \forall x \in S : f[x] \neq A}", \\
    \> "\NEW S", \\
    \> "\NEW f", "f \in [S -> \SUBSET\ S]", \\
    \> "T \DEF \{z \in S : z \notin f[z]\}", \\
    \> "\hide{\lnot \left(\exists A \in \SUBSET\ S : \forall x \in S : f[x] \neq A\right)}", \\
    \> "\s22 \DEF \forall x \in S : f[x] \neq T", \\
    \> "\hide{\lnot \left( \forall x \in S : f[x] \neq T \right)}", \\
    \> "\s31 \DEF \obl{\NEW x, x \in S ||- f[x] \neq T}", \\
    \> "\NEW x", "x \in S", \\
    \> "\hide{\lnot \left( f[x] \neq T \right)}", \\
    \> "\s41 \DEF \obl{x \in T ||- f[x] \neq T}", \\
    \> "x \in T" \\
    \> "||- f[x] \neq T" "\smash{\Bigr)}".
  \end{tabbing}
\end{quote}
%
Filtering its obligation (see Def.~\ref{defn:filter}) and expanding all definitions gives:
%
\begin{quote} \small
  \begin{tabbing}
    "\smash{\Bigl(}"
    \= "\NEW S", \\
    \> "\NEW f", "f \in [S -> \SUBSET\ S]", \\
    \> "\NEW x", "x \in S", \\
    \> "x \in \{z \in S : z \notin f[z]\} ||- f[x] \neq \{z \in S : z \notin f[z]\}" "\smash{\Bigr)}".
  \end{tabbing}
\end{quote}
%
In Isabelle/\tlaplus, this is the following lemma:
\begin{quote} \small
  \begin{tabbing}
    \texttt{lemma} \= "!! S." \\
                   \> "!! f.\ " \= " f \in [S -> \SUBSET\ S] ==>" \\
                   \>                 \> "\Bigl(!! x.\ \bigl\llbracket" \= "x \in S ;" \\
                   \> \> \> "x \in \{z \in S : z \notin f[z]\} \ \bigr\rrbracket
                         ==> f[x] \neq \{z \in S : z \notin f[z]\} \Bigr)".
  \end{tabbing}
\end{quote}



%%%% PLEASE DO NOT EDIT BELOW THIS LINE
\ifx\master\undefined
{\let\master\relax%%% -*- mode: LaTeX; TeX-master: "main.tex"; -*-

\ifx\master\undefined
\documentclass[a4paper]{easychair}
\usepackage{submission}
\begin{document}
{\let\master\relax %%% -*- mode: LaTeX; TeX-master: "main.tex"; -*-

\ifx\master\undefined
\documentclass[a4paper]{easychair}
\usepackage{submission}
\begin{document}
\fi
%%%% PLEASE DO NOT EDIT ABOVE THIS LINE

\title{A \tlaplus Proof System}

\titlerunning{A \tlaplus Proof System}

% \volumeinfo
% 	{P. Rudnicki, G. Sutcliffe} % editors
% 	{2}                         % number of editors
% 	{KEAPPA 2008}               % event
% 	{1}                         % volume
% 	{1}                         % issue
% 	{1}                         % starting page number


%  Alphabetically by surname
\author{
  Kaustuv Chaudhuri \\
  INRIA \\
  \and
  Damien Doligez \\
  INRIA \\
  \and
  Leslie Lamport \\
  Microsoft Research \\
  \and
  Stephan Merz \\
  INRIA \& Loria
}

\authorrunning{Chaudhuri, Doligez, Lamport, and Merz}

\maketitle

%%%% PLEASE DO NOT EDIT BELOW THIS LINE
\ifx\master\undefined
{\let\master\relax %%% -*- mode: LaTeX; TeX-master: "main.tex"; -*-

\ifx\master\undefined
\documentclass[a4paper]{easychair}
\usepackage{submission}
\begin{document}
{\let\master\relax \input{frontmatter}}
\fi
%%%% PLEASE DO NOT EDIT ABOVE THIS LINE

\bibliographystyle{plain}
\bibliography{submission}

%%%% PLEASE DO NOT EDIT BELOW THIS LINE
\ifx\master\undefined
\end{document}
\fi

% LocalWords:  tex Paxos
}
\end{document}
\fi

% LocalWords:  tex Rudnicki Sutcliffe KEAPPA Kaustuv Chaudhuri INRIA Doligez
% LocalWords:  Merz Loria
}
\fi
%%%% PLEASE DO NOT EDIT ABOVE THIS LINE

\bibliographystyle{plain}
\bibliography{submission}

%%%% PLEASE DO NOT EDIT BELOW THIS LINE
\ifx\master\undefined
\end{document}
\fi

% LocalWords:  tex Paxos
}
\end{document}
\fi
