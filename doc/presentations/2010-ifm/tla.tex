\section{Introductory example: Euclid's algorithm}


\begin{frame}
  \frametitle{Euclid's Algorithm}

  \begin{itemize}
  \item \tc{dkblue}{Euclid's algorithm in pseudo-code}

    \bigskip

    \hspace*{4em}
    \begin{minipage}{.5\linewidth}\small
    \begin{tabbing}
      \quad\=\quad\=\kill
      \textbf{variables} x = M, y = N\\
      \textbf{begin}\\
      \> \textbf{while} x $\neq$ y \textbf{do}\\
      \>\> \textbf{if} x$<$y\\
      \>\> \textbf{then} y := y-x\\
      \>\> \textbf{else} x := x-y\\
      \>\> \textbf{end if}\\
      \> \textbf{end while};\\
      \> \textbf{assert} $GCD$(M,N) = x\\
      \textbf{end}
    \end{tabbing}
    \end{minipage}

  \oo \tc{dkblue}{This is a legal PlusCal algorithm}

    \begin{itemize}
    \o embedded in a \tlaplus\ module defining $GCD$
    \o can be checked for fixed values of M and N
    \end{itemize}
  \end{itemize}
\end{frame}

\begin{frame}
  \frametitle{Euclid's Algorithm in \tlaplus\ (1/2)}

  \begin{itemize}
  \item \tc{dkblue}{We start by defining divisibility and $GCD$}

    \medskip

    \begin{tlablock}
      \begin{minipage}{.96\linewidth}
      \begin{nomodule}
        \topbar{Euclid}
        \EXTENDS\ Naturals\\[1mm]
        \(\begin{noj}
          d | q\ \deq\ \E k \in 1..q : q = k * d
          \quad\quad\quad\comment{definition of divisibility}\\
          Divisors(q)\ \deq\ \{d \in 1..q : d | q\}
          \quad\ \ \comment{set of divisors}\\
          Maximum(S)\ \deq\ \CHOOSE x \in S : \A y \in S : x \geq y\\
          GCD(p,q)\ \deq\ Maximum(Divisors(p) \cap Divisors(q))\\
          PosInteger\ \deq\ Nat \setminus \{0\}
        \end{noj}\)\\
        \midbar
      \end{nomodule}
      \end{minipage}
    \end{tlablock}

  \oo \tc{dkblue}{Standard mathematical definitions}

    \begin{itemize}
    \o \tlaplus\ module $Naturals$ defines basic operations on integers
    \o \tlaplus\ is based on untyped set theory
    \o module contains declarations, assertions, and definitions
    \end{itemize}

  \oo \tc{dkblue}{These definitions could go to a library module}
  \end{itemize}
\end{frame}

\begin{frame}
  \frametitle{Euclid's Algorithm in \tlaplus\ (2/2)}

  \begin{itemize}
  \item \tc{dkblue}{Now encode the algorithm and assert its correctness}

    \medskip

    \begin{tlablock}
      \begin{minipage}{.96\linewidth}
      \begin{nomodule}
        \CONSTANTS\ \ M, N\\
        \ASSUME\ \ $Positive\ \deq\ M \in PosInteger \land N \in PosInteger$\\
        \VARIABLES\ \ x, y\\[1mm]
        \(\begin{array}{@{}l@{\ \ }c@{\ \ }l}
          Init & \deq & x=M \land y=N\\
          Next & \deq &
          \begin{disj}
            \begin{conj}
              x<y\\
              y' = y-x \land x' = x
            \end{conj}\\
            \begin{conj}
              y<x\\
              x' = x-y \land y' = y
            \end{conj}
          \end{disj}\\
          Spec & \deq & Init \land \alw[Next]_{\seq{x,y}}
        \end{array}\)\\
        \midbar
        $Correctness\ \deq\ x=y \implies x = GCD(M,N)$\\[1mm]
        \THEOREM\ \ $Spec \implies \alw Correctness$\\
        \bottombar
      \end{nomodule}
      \end{minipage}
    \end{tlablock}


  \oo \tc{dkblue}{Algorithm represented by initial condition and next-state relation}

  \oo \tc{dkblue}{Correctness expressed as TLA formula}
  \end{itemize}
\end{frame}

\begin{frame}
  \frametitle{\tlaplus\ Modules}

  \begin{itemize}
  \item \tc{dkblue}{Specifications of \tlaplus\ are structured in modules}

    \begin{itemize}
    \o structured specifications: import existing modules via \EXTENDS
    \o \INSTANCE\ allows import with renaming, but we don't need it here
    \end{itemize}

  \oo \tc{dkblue}{Modules contain declarations, assertions, and definitions}

    \begin{itemize}
    \o declarations of \CONSTANTS\ and \VARIABLES
    \o assertions of facts via \ASSUME\ and \THEOREM\ (more later)
    \o main body of module: \alert{operator definitions}
    \end{itemize}

  \oo \tc{dkblue}{Levels of formulas and operators}

    \medskip

    \renewcommand{\arraystretch}{1.2}
    \quad{\small\begin{tabular}{l@{\qquad}l@{\qquad}l}
        \alert{constant} & only \CONSTANT\ symbols & \tc{dkgreen}{$Positive$}\\
        \alert{state} & allow \VARIABLE{}s & \tc{dkgreen}{$Init,\ Correctness$}\\
        \alert{action} & allow primed \VARIABLE{}s & \tc{dkgreen}{$Next$}\\
        \alert{temporal} & use temporal operators & \tc{dkgreen}{$Spec$}
    \end{tabular}}
  \end{itemize}
\end{frame}

\begin{frame}
  \frametitle{Verification of Euclid's Algorithm}

  \begin{itemize}
  \item \tc{dkblue}{Verification by model checking: \tlc}

    \begin{itemize}
    \o construct model by fixing concrete values for $M$ and $N$
    \o \tlc\ verifies that the $Correctness$ property is always true
    \o variation: verify correctness for all initial values in fixed interval
    \end{itemize}

\pause

  \oo \tc{dkblue}{Verification by theorem proving: \tlaps}

    \begin{itemize}
    \o need to strengthen correctness property to an \alert{inductive invariant}

       \medskip
       \begin{tlablock}[.7]
         InductiveInvariant\ \deq\ 
         \begin{conj}
           x \in PosInteger\\
           y \in PosInteger\\
           GCD(x,y) = GCD(M,N)
         \end{conj}
       \end{tlablock}
    \end{itemize}
  \end{itemize}
\end{frame}

\begin{frame}
  \frametitle{Underlying Data Properties}

  \begin{itemize}
  \item \tc{dkblue}{The proof relies on the following properties of $GCD$}

     \bigskip

     \begin{tlablock}
       \begin{array}{@{}l@{\ \ }c@{\ \ }l}
         \THEOREM\ \ GCDSelf & \deq &
         \begin{array}[t]{@{}l@{\ \ }l}
           \ASSUME & \NEW\ p \in PosInteger\\
           \PROVE  & GCD(p,p) = p
         \end{array}\vspace{2mm}\\ 
         \THEOREM\ \ GCDSymm & \deq &
         \begin{array}[t]{@{}l@{\ \ }l}
           \ASSUME & \NEW\ p \in PosInteger,\\
                   & \NEW\ q \in PosInteger\\
           \PROVE  & GCD(p,q) = GCD(q,p)
         \end{array}\vspace{2mm}\\
         \THEOREM\ \ GCDDiff & \deq &
         \begin{array}[t]{@{}l@{\ \ }l}
           \ASSUME & \NEW\ p \in PosInteger,\\
                   & \NEW\ q \in PosInteger,\\
                   & p<q\\
           \PROVE  & GCD(p,q) = GCD(p, q-p)
         \end{array}
       \end{array}
     \end{tlablock}

  \oo \tc{dkblue}{\tc{dkgreen}{$\ASSUME$ \ldots\ $\PROVE$} assertions are sequents in \tlaplus}

    \begin{itemize}
    \o could use formulas instead, but sequents are often easier to read
    \end{itemize}

  \oo \tc{dkblue}{We don't bother proving these properties here}
  \end{itemize}
\end{frame}

\begin{frame}
  \frametitle{Invariant Reasoning in \tlaplus}

  \begin{itemize}
  \item \tc{dkblue}{Establish an invariant in \tlaplus}

    \medskip

\begin{center}
    \qquad\(\color{red!75!black}\begin{array}{c}
      Init \implies Inv \quad Inv \land [Next]_v \implies Inv' \quad Inv \implies Cor\\
      \hline
      Init \land \alw[Next]_v \implies \alw Cor
    \end{array}\)
\end{center}

    \begin{itemize}
%    \o $J$ is an inductive invariant that implies $Inv$
    \o $Inv$ must imply $Cor$, be true initially, and preserved by every step
    \end{itemize}

\pause

  \o \tc{dkblue}{This rule can be stated as the following sequent}

     \medskip

     \qquad\begin{tlablock}[.7]
       \THEOREM\ \ Inv1\ \deq\ 
       \begin{array}[t]{@{}l@{\ \ }l}
         \ASSUME & Init \implies Inv,\\
                 & Inv \land [Next]_v \implies Inv',\\
                 & Inv \implies Cor\\
         \PROVE  & Init \land \alw[Next]_v \implies \alw Cor
       \end{array}
     \end{tlablock}

     \begin{itemize}
     \o \tlaps\ doesn't handle temporal logic yet
     \o but it can be used to establish the non-temporal hypotheses
     \end{itemize}
  \end{itemize}
\end{frame}

\begin{frame}
  \frametitle{Simple Proofs}

  \begin{itemize}
  \item \tc{dkblue}{Prove that \tc{dkgreen}{$InductiveInvariant$} implies \tc{dkgreen}{$Correctness$}}

    \medskip

    \qquad\begin{tlablock}
      \LEMMA\ \ InductiveInvariant \implies Correctness\\
      \only<1>{\alert{\PROOF\ \OBVIOUS}}
      \only<2->{\BY\ GCDSelf\ \DEFS\ InductiveInvariant, Correctness}
    \end{tlablock}

\pause



    \begin{itemize}
    \o definitions and facts must be cited explicitly for \tlaps\ to use them
    \o this helps keeping the size of proof obligations manageable
    \end{itemize}

\pause
  \oo \tc{dkblue}{Prove that \tc{dkgreen}{$Init$} implies \tc{dkgreen}{$InductiveInvariant$}}

    \medskip

    \qquad\begin{tlablock}
      \LEMMA\ \ Init \implies InductiveInvariant\\
      \BY\ Positive\ \DEFS\ Init, InductiveInvariant
    \end{tlablock}

  \oo \tc{dkblue}{These simple proofs are called \alert{leaf proofs}}
  \end{itemize}
  
\end{frame}

\begin{frame}
  \frametitle{Hierarchical Proofs}

  \begin{itemize}
  \item \tc{dkblue}{A non-leaf proof consists of a sequence of claims, ending with \QED}

  \oo \tc{dkblue}{Prove that \tc{dkgreen}{$Next$} preserves \tc{dkgreen}{$InductiveInvariant$}}

    \medskip

    \qquad\begin{tlablock}
      \LEMMA\ \ InductiveInvariant \land [Next]_{\seq{x,y}} \implies InductiveInvariant'\\
      \ps{1}{}\ \USE\ \DEFS\ InductiveInvariant, Next\\
\onslide<2->{
      \ps{1}{}\ \SUFFICES\ 
        \begin{array}[t]{@{}l@{\ \ }l}
          \ASSUME & InductiveInvariant, Next\\
          \PROVE  & InductiveInvariant'
        \end{array}\\
        \quad \PROOF\ \OBVIOUS
  }\\
\onslide<3->{
      \ps{1}{a.}\ \CASE\ x<y\\
      \ps{1}{b.}\ \CASE\ x>y}\\
\onslide<4->{
      \ps{1}{q.}\ \QED\\
      \quad  \quad \BY\ \ps{1}{a}, \ps{1}{b}}
    \end{tlablock}

    \begin{itemize}
    \o \only<1>{\USE\ \DEFS\ causes \tlaps\ to silently apply given definitions.}
       \only<2>{\SUFFICES\ restates the current claim -- trivial case $\UNCHANGED \seq{x,y}$}
       \only<3>{The two subcases will be proved subsequently.}
       \only<4>{The assertion follows from the cases and the definition of $Next$.}
    \end{itemize}
  \end{itemize}
\end{frame}



\begin{frame}[t]
  \frametitle{Hierarchical Proofs}

  \begin{itemize}
  \item \tc{dkblue}{Sublevels}

\medskip

    \qquad\begin{tlablock}
      {\color{gray}(...)} \\
      \ps{1}{a.}\ \CASE\ x<y\\
      \quad \ps{2}{1.} \ (y - x \in PosInteger)  \land \lnot(y < x)\\
      	\only<3->{\quad\quad \BY\ \ps{1}{a},\ SimpleArithmetic\ \DEF\ PosInteger}\\
  	\quad \ps{2}{2.} \  \QED\\
    	\only<2->{\quad\quad \BY\ \ps{1}{a}, \ps{2}{1},\ GCDDiff}\\
      \ps{1}{b.}\ \CASE\ x>y\\
 {\color{gray}(...)} 
    \end{tlablock}
    
    \medskip

\uncover<3->{
    \item $SimpleArithmetic$
    \begin{itemize}
    \o theorem from the standard module TLAPS.tla
    \o calls another back-end
    \o Cooper's algorithm for Presburger's arithmetic 
    \end{itemize}
}
  \end{itemize}
\end{frame}



%%% Local Variables: 
%%% mode: latex
%%% TeX-master: "tutorial"
%%% End: 
