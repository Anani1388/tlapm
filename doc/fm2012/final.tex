\let\master\relax
\documentclass[a4paper,draft]{llncs}
 
%&b&\textbf{#}&
%&c&\textsc{#}&
%&t&\texttt{#}&

%\listfiles

\usepackage{url}
\usepackage{xspace}

%% The tlatex package defines the tlatex environment.
%% The stuff in the tlatex environments in this paper was created
%% by tweaking the LaTeX output produced by TLATeX gets put.  
%% You should not muck with anything in a tlatex environment unless 
%% you're sure you know what you're doing.
\usepackage{tlatex}

%\pagestyle{plain}    %% remove for final version
%\raggedbottom

\title{\tlaplus\ Proofs}
	
%  Alphabetically by surname
\author{
   Denis Cousineau\inst{1} \and
   Damien Doligez\inst{2} \and
   Leslie Lamport\inst{3} \and
   Stephan Merz\inst{4} \and \\
   Daniel Ricketts\inst{5} \and
   Hern\'an Vanzetto\inst{4}
}

\authorrunning{Cousineau, Doligez, Lamport, Merz, Ricketts and Vanzetto}

\institute{
   Inria - Universit\'e Paris Sud, Orsay, France. \footnote{This work was partially funded by Inria-Microsoft Research Joint Centre, France.} \and
   Inria, Paris, France \and
   Microsoft Research, Mountain View, CA, U.S.A. \and
   Inria Nancy \& LORIA, Villers-l\`es-Nancy, France \and
   Department of Computer Science, University of California, San Diego, U.S.A.
}


\newcommand{\implies}{\Rightarrow}
\newcommand{\tlaplus}{\mbox{TLA\kern -.35ex$^+$}\xspace}
\newcommand{\PM}{PM\xspace}
\newcommand{\tlastring}[1]{\textsf{``#1''}}
\newcommand{\s}[1]{\ensuremath{\left\langle#1\right\rangle}}

\newenvironment{noj}{\begin{array}[t]{@{}l@{}}}{\end{array}}
\newenvironment{conj}{\begin{array}[t]{@{\mbox{$\land\ $}}l}}{\end{array}}
\newenvironment{disj}{\begin{array}[t]{@{\mbox{$\lor\ $}}l}}{\end{array}}


% \step{3}{4} produces step number <3>4
\makeatletter
\newcommand{\step}[2]{{\tlatex \@pfstepnum{#1}{#2}}}
\makeatother

% Some other definitions.  
% Spacing definitions for formatting the figures.
\def\S#1{\hspace*{#1em}}
\def\T#1{\hspace*{-#1pt}}

% The following defines \str{foo} to be the properly 
% typeset TLA+ string "foo".
\makeatletter \let\str=\@w \makeatother

% The display environment can be used to set off things like formulas
% and program statements
\newenvironment{display}{\begin{itemize}\item[]}{\end{itemize}}

% \proofrule{A}{B} produces:   A
%                             ---
%                              B
% \newcommand{\proofrule}[2]{\setlength{\arrayrulewidth}{.6pt}%
%   {\ensuremath{\begin{array}[t]{@{}c@{}}%
%        \begin{array}[t]{@{}l@{}} #1\raisebox{-.1em}{\strut}\end{array}\\
%        \hline \raisebox{.1em}{\strut}#2\end{array}}}}


\begin{document}

\maketitle

%\ifdraft
%\begin{center}
%\large\today
%\end{center}
%\fi

\begin{abstract}
  \tlaplus is a specification language based on standard set theory and temporal
  logic that has constructs for hierarchical proofs. We describe how to write
  \tlaplus proofs and check them with TLAPS, the \tlaplus Proof System. We use
  Peterson's mutual exclusion algorithm as a simple example
  and show how TLAPS and the Toolbox (an IDE for \tlaplus) help
  users to manage large, complex proofs.
\end{abstract}



\section{Introduction}

\tlaplus~\cite{lamport03tla} is a specification language originally designed for
specifying concurrent and distributed systems and their properties.
It is based on Zermelo-Fraenkel set theory for modeling data structures and on
the linear-time temporal logic TLA for specifying system executions and their
properties.
% Specifications and properties are written as formulas of TLA, a linear-time
% temporal logic. \tlaplus\ is based on TLA and Zermelo-Fraenkel set theory with
% the axiom of choice; it also adds a module system for structuring
% specifications. 
More recently, constructs for writing proofs have been added to
\tlaplus, following a proposal for presenting rigorous hand proofs in a
hierarchical style~\cite{lamport:howtoprove-21century}.

In this paper, we present the main ideas that guided the design of the
proof language and its implementation in TLAPS, the \tlaplus proof
system~\cite{chaudhuri:tlaps,tlaps}.  The proof language and TLAPS have been
designed to be independent of any particular theorem prover.  All
interaction takes place at the level of \tlaplus.  Users need know
only what sort of reasoning TLAPS's backend provers tend to be good
at---for example, that SMT solvers excel at arithmetic.  This knowledge
is gained mostly by experience.

TLAPS has a \emph{Proof Manager} (\PM) that transforms a proof into
individual proof obligations that it sends to backend provers.
Currently, the main backend provers are Isabelle/\tlaplus, an encoding
of \tlaplus as an object logic in Isabelle~\cite{wenzel:isabelle},
Zenon~\cite{bonichon07lpar}, a tableau prover for classical
first-order logic with equality, and a backend for SMT solvers.
% The \PM\ also provides an implementation of Cooper's
% algorithm for Presburger arithmetic.  
Isabelle serves as
the most trusted backend prover, and when possible, we expect backend provers to
produce a detailed proof that is checked by Isabelle. This is currently
implemented for the Zenon backend.

TLAPS has been integrated into the \tlaplus Toolbox, an IDE (Integrated Development
Environment) based on Eclipse for writing \tlaplus specifications and
running the \tlaplus tools on them, including the TLC model checker. The Toolbox
provides commands to hide and unhide parts of a proof, allowing a user to focus
on a given proof step and its context. It is also invaluable to be able to run
the model checker on the same formulas that one reasons about.

We explain how to write and check \tlaplus\ proofs,
using a tiny well-known example: a proof that Peterson's
algorithm~\cite{peterson:myths} implements
mutual exclusion. We start by writing the algorithm in
PlusCal~\cite{lamport:pluscal}, an algorithm language that is based on the
expression language of \tlaplus. The PlusCal code is translated to a \tlaplus\
specification, which is what we reason about. Section~\ref{sec:proving}
introduces the salient features of the proof language and of TLAPS with the
proof of mutual exclusion. Liveness of Peterson's
algorithm (processes eventually enter their critical section) can also be
asserted and proved with \tlaplus. However, liveness reasoning makes full use of
temporal logic, and TLAPS cannot yet check temporal logic proofs.

% The following paragraph is rather wimpy.  It could do with a bit
% fleshing out--especially the last sentence.
Section~\ref{sec:real-proofs} indicates the features that make \tlaplus, TLAPS,
and the Toolbox scale to realistic examples. A concluding section summarizes
what we have done and our plans for future work.



\section{Modeling Peterson's Algorithm In \tlaplus}
\label{sec:peterson}

Peterson's algorithm %\cite{peterson:myths} 
is a classic, very simple
two-process mutual exclusion algorithm.  We specify the algorithm in
\tlaplus and prove that it satisfies mutual exclusion: no two
processes are in their critical sections at the same
time.\footnote{The \tlaplus module containing the
  specification and proof as well as an extended version
  of this paper are accessible at the TLAPS Web page~\cite{tlaps}.}

% \subsection{From PlusCal To \tlaplus}
% \label{sec:pluscal}

A representation of Peterson's algorithm in the PlusCal algorithm language is
shown on the left-hand side of Figure~\ref{fig:the-algorithm}. The two processes
are named $0$ and $1$; the PlusCal code is embedded in a \tlaplus module that
defines an operator $Not$ so that $Not(0)=1$ and $Not(1)=0$.
% %
% \begin{display}
% \begin{tlatex}
% \@x{ Not ( i ) \.{\defeq}\, {\IF} i \.{=} 0 \.{\THEN} 1 \.{\ELSE} 0}
% \end{tlatex}
% \end{display}

\begin{figure}[tb]
  \begin{minipage}{.42\linewidth}
    \newlength{\labelsdim}
    \settowidth{\labelsdim}{$a3a$:}
    \newcommand{\makelab}[1]{\makebox[\labelsdim][r]{$#1$: }}
    \begin{tabbing}
    \texttt{-{}-}\textbf{algorithm} Peterson \{ \\
    \S{1}\=\+
      \textbf{variables}\\
        \S{1}\=\+ $flag = [i \in \{0, 1\} \mapsto \FALSE]$,\\
                  $turn = 0$; \- \\
      \textbf{process} $(proc \in \{0,1\})$ \{\\
      \S{1}\=\+ \makelab{a0} \textbf{while} $(\TRUE)$ \{ \\
         \makelab{a1}\S{1}\=   $flag[self] := \TRUE$; \\
         \makelab{a2}\>   $turn := Not(self)$; \\
         \makelab{a3a}\>  \textbf{if} $(flag[Not(self)])$\\
                      \>  \S{2} \= \{\textbf{goto} $a3b$\}\\
                      \>  \textbf{else}\> \{\textbf{goto} $cs$\} ; \\
         \makelab{a3b}\>  \textbf{if} $(turn = Not(self))$ \\
                      \>        \> \{\textbf{goto} $a3a$\}\\
                      \>  \textbf{else} \> \{\textbf{goto} $cs$\} ; \\
         \makelab{cs}\>   \textbf{skip};  $\backslash*$ critical section \\
         \makelab{a4}\>   $flag[self] := \FALSE$; \\
         \hspace*{\labelsdim}\ \}  $\backslash*$  end while \- \\
        \} $\backslash*$  end process \- \\
      \S{.5}\} $\backslash*$  end algorithm 
    \end{tabbing}
  \end{minipage}
  \hfill
  \begin{minipage}{.52\linewidth}
    \renewcommand{\arraystretch}{1.2}
    \textsc{variables} $flag, turn, pc$

    \smallskip

    $vars \defeq \s{flag, turn, pc}$

    \smallskip

    $Init \defeq
     \begin{conj}
       flag = [i \in \{0,1\} \mapsto \FALSE]\\
       turn = 0\\
       pc = [self \in \{0,1\} \mapsto \tlastring{a0}]
     \end{conj}$

    \smallskip

    $a3a(self) \defeq$\\
    \mbox{\ \ }$\begin{conj}
      pc[self] = \tlastring{a3a}\\
      \begin{noj}
        \IF flag[Not(self)]\\
        \textsc{then } pc' = [pc \EXCEPT ![self] = \tlastring{a3b}]\\
        \textsc{else } pc' = [pc \EXCEPT ![self] = \tlastring{cs}]
      \end{noj}\\
      \UNCHANGED \s{flag, turn}
    \end{conj}$

    \medskip

    $\backslash*$ remaining actions omitted

    \medskip

    $proc(self) \defeq a0(self) \lor \ldots \lor a4(self)$

    \smallskip

    $Next \defeq \E\, self \in \{0,1\}: proc(self)$

    \smallskip

    $Spec \defeq Init \land \Box[Next]_{vars}$
  \end{minipage}
\caption{Peterson's algorithm in PlusCal (left) and in \tlaplus (excerpt, right).}
\label{fig:the-algorithm}
\end{figure}


The \textbf{variables} statement declares the variables and their initial
values. For example, the initial value of $flag$ is an array such that
$flag[0]=flag[1]=\FALSE$. (Mathematically, an array is a function; the
\tlaplus\ notation $[x \in S \mapsto e]$ 
 for writing functions
is similar to a lambda expression.)
To specify a multiprocess
algorithm, it is necessary to specify what its atomic actions are.  In
PlusCal, an atomic action consists of the execution from one label to the
next.  With this brief explanation, the reader should be able to
figure out what the code means.

A 
%The PlusCal 
translator, normally called from the Toolbox,
generates a \tlaplus
specification from the PlusCal code. We illustrate the
structure of the \tlaplus translation in the right-hand part of
Figure~\ref{fig:the-algorithm}. 
% \footnote{For clarity of presentation, we have simplified the translation
%   slightly by ``in-lining'' a definition. The proof we develop works for the
%   unmodified translation if we add a global declaration that causes the
%   definition to be expanded throughout the proof.}
%
The heart of the \tlaplus specification consists of the predicates
$Init$ describing the initial state and $Next$,
which represents the next-state relation. 

The PlusCal translator adds a variable $pc$ to record the control state of each
process.
% For example,
% control in process $i$ is at $cs$ iff $pc[i]$ equals the string \str{cs}.
The meaning of formula $Init$ in the figure is straightforward. The formula
$Next$ is the disjunction of the two formulas $proc(0)$ and $proc(1)$, which are
in turn defined as disjunctions of formulas corresponding to the atomic steps of
the \textbf{process}. 
% As an example, we show the definition of the formula $a3a(self)$. 
In these formulas, unprimed variables
refer to the old state and primed variables to the new state.
% The reader should be able to figure out the meaning of the \tlaplus\ 
% notation and of formula $Next$ by comparing these seven definitions
% with the corresponding PlusCal code.
%
The temporal formula $Spec$ is the complete specification. It characterizes
behaviors ($\omega$-sequences of states) that start in a state satisfying
$Init$ and where every pair of successive states either satisfies $Next$ or else
leaves the values of the tuple $vars$ unchanged.\footnote{``Stuttering steps''
  are allowed in order to make refinement simple~\cite{lamport:what-good}.}


% \subsection{Validation Through Model Checking}
% \label{sec:model-checking}

Before trying to prove that the algorithm is correct, we use TLC,
the \tlaplus model checker, to check it for errors.
% We first instruct the Toolbox to have TLC check for
% ``execution errors''.\footnote{The translation is a
%   temporal logic formula, so there is no obvious definition of an
%    execution error.  An execution error occurs in a
%   behavior if whether or not the behavior satisfies the formula is not
%   specified by the semantics of \tlaplus---for example, because
%   the semantics do not specify whether or not 0 equals $\FALSE$.}
% What are type errors in typed languages are one source of execution
% errors in \tlaplus.
%
The Toolbox runs TLC on a model of a \tlaplus specification.  A model
usually assigns particular values to specification constants, such as
the number of processes. It can also restrict the set of states
explored, which is useful if the specification allows an infinite number
of reachable states.
% For this trivial example, there are no constants
% to specify and only 146 reachable states.  TLC finds no execution
% errors.
%
TLC easily verifies that the two processes can never 
both be at label $cs$ by checking that the following formula is
an invariant (true in all reachable states):
% 
% We check if the algorithm actually satisfies mutual exclusion, i.e.\ if the two
% processes can never both have control
% at label $cs$:
%
\[
  MutualExclusion \defeq 
  (pc[0] \neq \tlastring{cs}) \lor (pc[1] \neq \tlastring{cs})
\]
% 
% TLC reports that this predicate is indeed an invariant of the model, that is,
% that it holds in all reachable states. 
Peterson's
algorithm is so simple that TLC can check all possible
executions. For more interesting algorithms that have parameters (such as
the number of processes) and perhaps
an infinite set of reachable states, TLC cannot exhaustively
verify all executions, and correctness can only be proved deductively. Still, TLC is
invaluable for catching errors, and
% in the algorithm or its formal model: 
it is much easier to run TLC than to write a
formal proof.

% The following section describes a deductive correctness proof of Peterson's
% algorithm in \tlaplus. Proofs of more interesting algorithms follow the same
% basic structure, but they are longer and require more complicated reasoning.
% Section~\ref{sec:real-proofs} describes how \tlaplus\ proofs scale to larger
% algorithms.


\section{Proving Mutual Exclusion For Peterson's Algorithm}
\label{sec:proving}

\begin{figure}[t]
  \centering
  \(\begin{noj}
    \THEOREM Spec \implies \Box MutualExclusion\\
    \step{1}{1.}\ Init \implies Inv\\
    \step{1}{2.}\ Inv \land [Next]_{vars} \implies Inv'\\
    \step{1}{3.}\ Inv \implies MutualExclusion\\
    \step{1}{4.}\ \QED
  \end{noj}\)
  \caption{The high-level proof.}
  \label{fig:high-level}
\end{figure}

The assertion that Peterson's algorithm implements mutual exclusion is
formalized in \tlaplus as the theorem 
in Figure~\ref{fig:high-level}. The standard method of proving this invariance
property is to find an inductive invariant $Inv$ 
such that the steps \step{1}{1}--\step{1}{3} of Figure~\ref{fig:high-level} are
provable.

\tlaplus proofs are hierarchically structured and are generally written
top-down. Each proof in the hierarchy ends with a \QED step that asserts the
proof's goal. We usually write the \QED step's proof before the proofs of
the intermediate steps. The \QED step follows easily from steps
\step{1}{1}--\step{1}{3} by standard proof rules of temporal logic. However,
TLAPS does not yet handle temporal reasoning, so we omit that step's proof.
When temporal reasoning is added to TLAPS, we expect it easily to check
such a trivial proof.

\begin{figure}[b]
\(\begin{noj}
  TypeOK \defeq
  \begin{conj}
    pc \in [\:\{0,1\} \rightarrow \{\str{a0}, \str{a1}, \str{a2}, \str{a3a}, \str{a3b}, \str{cs}, \str{a4}\}\:]\\
    turn \in \{0,1\}\\
    flag \in [\:\{0,1\} \rightarrow \BOOLEAN]
  \end{conj}\\[27pt]
  I \defeq \forall i \in \{0,1\}:\\
  \hspace*{3.8em}\begin{conj}
    pc[i] \in \{\str{a2}, \str{a3a}, \str{a3b}, \str{cs}, \str{a4}\} \implies flag[i]\\
    pc[i] \in \{\str{cs}, \str{a4}\} \implies
    \begin{conj}
      pc[Not(i)] \notin \{\str{cs}, \str{a4}\}\\
      pc[Not(i)] \in \{\str{a3a}, \str{a3b}\} \implies turn = i
    \end{conj}
  \end{conj}\\[27pt]
  Inv \defeq TypeOK \land I
\end{noj}\)
\caption{The inductive invariant.}
\label{fig:inv}
\end{figure}

Figure~\ref{fig:inv} defines the inductive invariant $Inv$ as the conjunction of
two formulas.
(A definition must precede its use, so the definition of $Inv$ appears
in the module before the proof.) 
The first, $TypeOK$, asserts simply that the
values of all variables are elements of the expected sets.  (The
expression $[S\rightarrow T]$ is the set of all functions whose domain is
$S$ and whose range is a subset of $T$.)  In an untyped logic like that of
\tlaplus, almost any inductive invariant must assert type correctness.
The second conjunct, $I$, is the interesting one that explains why
Peterson's algorithm implements mutual exclusion. We again use TLC to check that
$Inv$ is indeed an invariant. In our simple example, TLC can even check that
$Inv$ is inductive, by checking that it is an (ordinary) invariant of the
specification
 \,\mbox{$Inv \land \Box[Next]_{vars}$}\,,
obtained from $Spec$ by replacing the initial condition by $Inv$.
% In most real examples, TLC can at best check an inductive invariant on
% a tiny model---one that is too small to gain any confidence that
% it really is an inductive invariant.  However, TLC can still often find
% simple errors in an inductive invariant.

We now prove steps \step{1}{1}--\step{1}{3}. We can prove them in any order; let
us start with \step{1}{1}. This step follows easily from the definitions, and
the following leaf proof is accepted by TLAPS:
\begin{display}
\textsc{by def} $Init$, $Inv$, $TypeOK$, $I$
\end{display}
%
TLAPS will not expand definitions unless directed to so.
In complex proofs, automatically expanding definitions often leads to
formulas that are too big for provers to handle. 
Forgetting to expand some definition is a common mistake.
If a proof does not succeed, the
Toolbox 
% reports failure and 
displays the exact proof obligation that it passed
to the prover. It is then usually easy to see which definitions need to be invoked.


Step \step{1}{3} is proved the same way, by simply expanding the definitions of
$MutualExclusion$, $Inv$, $I$, and $Not$. We next try the same
technique on \step{1}{2}. A little thought shows that we have to tell TLAPS to
expand all the definitions in the module up to and including the definition of
$Next$, except for the definition of $Init$. Unfortunately, when we direct TLAPS
to prove the step, it fails to do so, reporting a 65-line proof obligation.

TLAPS uses Zenon and Isabelle as its default backend provers.
% , first
% trying Zenon and then trying Isabelle if Zenon fails to find a proof.
% (TLAPS can be directed to use Isabelle to check Zenon's
% proofs, but one would do that only as a last step after the entire
% proof has been checked.)
However, TLAPS also includes an SMT solver backend~\cite{merz:smt-tlaps}
that is capable of
handling larger ``shallow'' proof obligations---in particular, ones
that do not contain significant quantifier reasoning.  We instruct
TLAPS to use the SMT backend when proving the current step by writing
\begin{display}
\textsc{by} SMT \textsc{def} \ldots
\end{display}
The backend translates the proof obligation to the input language of SMT
solvers. 
% It then calls an SMT solver (CVC3 by default, variants of the backend
% exist for Yices and Z3). 
In this way, step \step{1}{2} is proved in a few seconds.
For sufficiently complicated algorithms, an SMT solver will not be able
to prove inductive invariance as a single obligation.  Instead, the proof will
have to be hierarchically decomposed.  We illustrate how this is done
by writing a proof of \step{1}{2} that can be checked using only the
Zenon and Isabelle backend provers.  

\begin{figure}[tb]
  \centering
  \(\begin{noj}
    \step{1}{2}.\ Inv \land [Next]_{vars} \implies Inv'\\
    \quad\begin{noj}
      \step{2}{1.}\ \SUFFICES \ASSUME Inv, Next\ \PROVE Inv'\\
      \step{2}{2.}\ TypeOK'\\
      \step{2}{3.}\ I'\\
      \quad\begin{noj}
        \step{3}{1.}\ \SUFFICES \ASSUME \NEW j \in \{0,1\}\ \PROVE I!(j)'\\
        \step{3}{2.}\ \PICK i \in \{0,1\}: proc(i)\\
        \step{3}{3.}\ \CASE i=j\\
        \step{3}{4.}\ \CASE i \neq j\\
        \step{3}{5.}\ \QED
      \end{noj}\\
      \step{2}{4.}\ \QED
    \end{noj}
  \end{noj}\)
%
\caption{Outline of a hierarchical proof of step \step{1}{2}.}
\label{fig:hierarchical-proof}
\end{figure}

The outline of a hierarchical proof of step \step{1}{2} appears in
Figure~\ref{fig:hierarchical-proof}. 
% All leaf steps can be proved using one-line
% proofs that cite available assumptions and expand necessary definitions. 
The proof
introduces more elements of the \tlaplus proof language that we now explain.

A \SUFFICES step allows a user to introduce an auxiliary assertion, from which
the current goal can be proved. For example, step \step{2}{1} reduces the proof
of the implication asserted in step \step{1}{2} to assuming predicates $Inv$ and
$Next$, and proving $Inv'$. In particular, this step establishes that the
invariant is preserved by stuttering steps that leave the tuple $vars$ unchanged. Steps
\step{2}{2} and \step{2}{3} establish the two conjuncts in the definition of
$Inv$. Whereas \step{2}{2} can be proved directly by Isabelle, \step{2}{3} needs
some more interaction. 

Following the definition of predicate $I$ as a universally quantified formula,
we introduce in step \step{3}{1} a new variable $j$, assume that $j \in
\{0,1\}$, and prove $I!(j)'$, which denotes the body of the universally
quantified formula, with $j$ substituted for the bound variable, and with primed
copies of all state variables. Similarly, step \step{3}{2} introduces variable
$i$ to denote the process that makes a transition, following the definition of
$Next$ (which is assumed in step \step{2}{1}). Even after this elimination of
two quantifiers, Isabelle and Zenon cannot prove the goal in a single
step. The usual way of decomposing the proof is to reason separately
about each atomic action $a0(i)$, \ldots, $a4(i)$. However,
Peterson's algorithm is simple enough that we can just split the proof
into the two cases $i=j$ and $i \neq j$ with steps \step{3}{3} and
\step{3}{4}.  Isabelle and Zenon can now prove all the steps.


\section{Writing Real Proofs}
\label{sec:real-proofs}

% We have described how one writes and checks a \tlaplus proof of 
Peterson's algorithm is a tiny example.
Some larger case studies have been carried out using the
system~\cite{lamport:byzantine-paxos,lu:pastry,parno:memoir}. Several
features of TLAPS and its Toolbox interface help in coping with the
complexity of large proofs.


\subsection{Hierarchical Proofs And The Proof Manager}
\label{sec:hierarchy}

Hierarchical structure is the key to managing complexity.  \tlaplus's
hierarchical and declarative proof language enables a user to keep
decomposing a complex proof into smaller steps until the steps become
provable by one of the backend provers.  In logical terms, proof steps
correspond to natural-deduction sequents that must be proved
in the current context.  The Proof Manager tracks the
context, which is modified by non-leaf proof steps.  For leaf proof
steps, it sends the corresponding sequent to the backend provers, and
records the result of the step's proof that they report.


Proof obligations are independent of one another, so
users can develop proofs in any order and work on different proof
steps independently.  
% This permits them to concentrate on the part of a planned proof that is most likely
% to be wrong and require changes to other parts.  
The Toolbox makes it easy to instruct TLAPS to check the proof of
everything in a file, of a single theorem, or of any step in the proof
hierarchy.
% It displays every obligation whose proof fails or
% is taking too long; in the latter case the user can cancel the proof.
% Clicking on the obligation shows the part of the proof that generated
% it.
Its editor helps reading and writing
large proofs, providing commands that
show or hide subproofs.
% Commands to hide a proof
% or view just its top level aid in reading a proof.  A
% command that is particularly useful when writing a subproof is one
% that hides all preceding steps that cannot be used in that subproof
% because of their positions in the hierarchy.
Although some other interactive proof systems offer hierarchical
proofs, we do not know of other systems that provide the Toolbox's
abilities to use that structure to aid in reading and writing proofs
and to prove steps in any order.

% \tlaplus's 
Hierarchical proofs are much better than conventional 
lemmas for handling complexity.
% for structuring complex proofs 
In a \tlaplus proof, each step with a non-leaf proof is
effectively a lemma.  One typical 1100-line invariance
proof~\cite{lamport:byzantine-paxos}
contains 100 such steps.  A conventional linear proof with 100 lemmas
would be impossible to read.
% \tlaplus proofs use lemmas only when they will make the entire proof
% simpler---for example, if a result is used in two different theorems.

Unlike most interactive proof assistants~\cite{wiedijk:provers}, TLAPS
is independent of any specific backend prover.  There is no way for a
user to indicate how available facts should be used by backends.
\tlaplus proofs are therefore less sensitive to changes in any
prover's implementation.


\subsection{Fingerprinting: Tracking The Status Of Proof Obligations}
\label{sec:fingerprinting}

During proof development, a user repeatedly modifies the proof
structure or changes details of the specification.  By default, TLAPS
does not re-prove an obligation that it has already proved---even if
the proof has been reorganized.  It can also show the user the impact
of a change by indicating which parts of the existing proof must be
re-proved.

The Proof Manager computes a \emph{fingerprint} of every obligation, 
which it
stores, along with the obligation's status, in a separate file.
% Technically, a proof obligation is canonically represented as a lambda term, 
% with bound variables
% replaced by de Bruijn indices~\cite{deBruijn72} such that their actual names in the
% \tlaplus proof are irrelevant. The context is minimized by erasing symbols and
% hypotheses that are not used in the step. 
The fingerprint is a compact canonical representation of the 
obligation and the relevant part of its context.
% of the resulting term, which is therefore insensitive
% to structural modifications of the proof context that do not affect the
% obligation's logical validity.
%
The Toolbox displays the proof status of each step, indicating by
color whether the step has been proved or some obligation in its proof
has failed or been omitted.
% (The user can control what information is displayed.)  
% Looking up an obligation's status takes little time, so the user can tell TLAPS
% to re-prove a step or a theorem even if only a small part of the proof has
% changed; TLAPS will recognize any obligation that has not changed and
% will not attempt to prove it anew.
% There is also a check-status command that displays the proof status without
% actually launching any proofs. 
% One can stop a partially-complete proof, restart it
% months later, and use this command to quickly retrieve the status of all proofs.
The only other proof
assistant that we know to offer a mechanism comparable to our fingerprinting
facility is the KIV system~\cite{balser:kiv}.

% An incident that occurred in the Byzantine Paxos proof reveals the
% advantages of our method of writing proofs.  The third author wrote
% the safety proof primarily as a way of debugging TLAPS, spending a
% total of several weeks over several months on it.  Later, when writing a
% paper about the algorithm, he discovered that it did not satisfy the
% desired liveness property, so it had to be modified.  He changed the
% algorithm, fixed minor bugs found by TLC, and reproved the safety
% property---all in a day and a half, with about 12 hours of
% actual work.  He was able to do it that fast because of the
% hierarchical proof structure, TLAPS's fingerprinting mechanism (about
% 3/4 of the proof obligations in the new proof had already been
% proved), and the Toolbox's aid in managing the proof. 
% Even with
% this help, it would have taken much
% longer if the bugs found by TLC had had
% to be found and fixed while writing the
% proof.


% \section{Related Work}
% \label{sec:related}

% We have designed the \tlaplus proof system as a platform for interactively
% verifying concurrent and distributed algorithms. Unlike
% most interactive proof assistants~\cite{wiedijk:provers}, TLAPS has been
% designed around a declarative proof language that is independent of any specific proof
% backend. \tlaplus proofs indicate what facts are needed to prove a certain
% result, but they do not specify precisely how the backend provers should use
% these facts. % (e.g., for forward or backward chaining, or for rewriting). 
% Although this lack of fine control can frustrate users who are intimately
% familiar with the inner workings of a particular prover,
% % and know how to apply it in the most efficient manner.
% declarative proofs are less dependent on specific backend provers and
% less sensitive to changes in their implementation.

% We write complex proofs by hierarchically structuring their logic. 
% % -- SM: I think we've said this before, but if you believe it is important to
% % -- reiterate the point, reinsert the following sentence.
% % Unlike proofs using conventional lemmas,
% % the syntactic structure of a \tlaplus proof can display its logical structure.
% The graphical user interface provides commands that support hierarchical proofs
% by allowing a user to zoom in on the current context and by supporting
% non-linear proof development. Although some other interactive proof systems such
% as Mizar~\cite{trybulec:mizar} and Isabelle/Isar~\cite{wenzel:isar} also offer
% hierarchical proofs, to the best of our knowledge these systems do not provide
% the Toolbox's abilities to use that structure to aid in reading and writing
% proofs and to prove individual steps in any order---facilities that we find
% crucial in developing and managing large proofs. The only other proof
% assistant that we know to offer a mechanism comparable to our fingerprinting
% facility is the KIV system~\cite{balser:kiv}.

% The Rodin toolset supporting the Event-B formal method~\cite{abrial:rodin}
% shares several aspects with TLAPS: 
% Event-B and \tlaplus are both based on set theory, both
% emphasize refinement as a way to structure formal developments, and Rodin and
% TLAPS mechanize proofs of safety properties with the help of different backend
% provers. Unlike with Event-B models, the structure of \tlaplus specifications is
% not fixed: any \tlaplus formula can be considered as a system specification or a
% property, and TLAPS does not impose a structure on invariant or refinement
% proofs. The interactive prover of Rodin is based on conventional tactic scripts
% rather than a declarative proof language.

% Provers designed for program verification such as VCC~\cite{cohen:vcc} or
% Why~\cite{herms:certified} target low-level source code rather
% than high-level specifications of algorithms. They are based on generators of
% verification conditions corresponing to programming constructs, that are
% discharged by invoking powerful automatic provers. User interaction is
% essentially restricted to the choice of suitable program annotations.



\section{Conclusion}
\label{sec:conclusion}

The proof of Peterson's algorithm illustrates the main constructs of the
hierarchical and declarative \tlaplus proof language. The algorithm is so simple
that we had to eschew the use of the SMT solver backend so we could write a
nontrivial proof. Section~\ref{sec:real-proofs} explains why TLAPS, used with
the \tlaplus Toolbox, can handle more complex algorithms and specifications.

A key feature of TLAPS is its use of multiple backend provers.
Different proof techniques, such as resolution, tableau methods,
rewriting, and SMT solving offer complementary strengths.  Future
versions of TLAPS will probably support additional backend provers.
Because multiple backends raise concerns about soundness, TLAPS
provides the option of having Isabelle certify proof traces produced
by backend provers; and this has been implemented for Zenon.  Still,
it is much more likely that a proof is meaningless because of an error
in the specification than that it is wrong because of an error in a
backend.  Soundness also depends on parts of the proof manager.

We cannot overstate the importance of having TLAPS integrated with the
other \tlaplus tools---especially the TLC model checker.  Finding
errors by running TLC on finite instances of a specification is much
faster and easier than discovering them when writing a proof.
%
% Users check the exact same specifications that
% appear in their proofs.  Less obvious is how useful it is that TLC can
% evaluate \tlaplus formulas.  When verifying a system, we don't
% want to prove well-known mathematical facts; we want to assume them.
% However, it is easy to make a mistake in formalizing even simple
% mathematics, and assuming the truth of an incorrect formula can lead
% to an incorrect proof.  TLC can usually check a formula on a large
% enough model to make us confident that our formalization of a correct
% mathematical result is indeed correct.
%
Also, verifying an algorithm or system may require standard
mathematical results.  For example, the correctness of a distributed
algorithm might depend on known facts about graphs.  Engineers want to
assume such results, not prove them.  However, it is easy to make a
mistake when formalizing mathematics.  TLC can check the exact
\tlaplus formulas assumed in a proof (on finite instances), greatly reducing the chance of
introducing an unsound assumption.

We are actively developing TLAPS. 
Our main short-term objective is to add support for temporal reasoning. 
% It is not obvious how best to extend natural deduction to temporal logic. 
We have designed a smooth extension of the existing proof language to
sequents containing temporal formulas.  We also plan to improve
support for standard \tlaplus data structures such as sequences.



% There's a bug caused by an intereaction between BibTeX and the \url
% command that is fixed by the following
%
%% File url.sty  created 25 Feb Mar 1996 by Leslie Lamport
%%
%% \url{ARG} -- typesets ARG as a URL, by: 
%%      * Using a \tt declaration.
%%      * Making  % ~ $ # & _ ^ and \  act like ordinary letters. $
%%      * Allowing a line break after each "." and before each  
%%        "/" and "//".
%%    It also allows a line break immediately before and after
%%    ARG, so you can allow a line break between A and RG by
%%    writing \url{A}\url{RG} instead of \url{ARG}.
\makeatletter
\newcommand{\realslash}{/}
\begingroup
\catcode`\/\active
\catcode`\.\active
\catcode`:\active
\gdef\urlslash{\@ifnextchar/{\doubleslash}{\discretionary{}{}{}\realslash}}
\gdef\urlend#1{\let/\urlslash\let.\urldot
                 \discretionary{}{}{}#1\discretionary{}{}{}\endgroup}
\endgroup
\newcommand{\urldot}{.\discretionary{}{}{}}
\renewcommand{\url}{\begingroup\urlbegin}
\newcommand{\urlbegin}{%\catcode`\%12\relax
                       \catcode`\~12\relax
                       \catcode`\#12\relax
                       \catcode`\$12\relax
                       \catcode`\&12\relax
                       \catcode`\_12\relax
                       \catcode`\^12\relax
                       \catcode`\\12\relax
                       \catcode`\/\active
                       \catcode`\.\active
                       \tt
                       \urlend}
\newcommand{\doubleslash}[1]{\discretionary{}{}{}//}

\makeatother 

\bibliographystyle{abbrv}
\bibliography{submission}
\end{document}
