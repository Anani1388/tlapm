\documentclass{article}
\usepackage{amsmath, amssymb}
\usepackage{minipage}
\usepackage{url}
\usepackage[left=2cm, right=2cm, top=2cm, bottom=2cm]{geometry}

\newcommand{\code}[1]{\mathsf{#1}}

\title{TLA PM -- Quick FAQ}
\author{Jael Kriener}

\begin{document}
\maketitle



\section{Introduction}
This is meant to be a quick reference guide to help the reader 
develop proofs. The code-patterns below will work in specific 
situations. There are other ways of writing the same things in 
TLA+ and the keywords used below can be used in other situations.
So please take this as a starting point, and refer to the TLA+ Guide\footnote{ 
\url{http://research.microsoft.com/en-us/um/people/lamport/tla/tla2-guide.pdf}}
and Chapter 10 of the Hyperbook\footnote{\url{http://research.microsoft.com/en-us/um/people/lamport/tla/hyperbook.html}} to go in further breadth and depth.





%%%%%%%%%%%%%%%%%%%%%%%%%%%%%%%%%%%%%%%%%%%%%%%%%%%%%%%%%%%%%%%%%%%%%%%
%%%%%%%%%%%%%%            HOW DO I PROVE ?            %%%%%%%%%%%%%%%%%
%%%%%%%%%%%%%%%%%%%%%%%%%%%%%%%%%%%%%%%%%%%%%%%%%%%%%%%%%%%%%%%%%%%%%%%
\section{How Do I Prove ... ?}

%%%%%%%%%%%%%%%%%%%%%%%%%%%%%%%%%%%%%%%%%%%%%%%%%%%%%%%%%%%%%%%%%%%%%%%
\subsection{$\code{A \wedge B}$}
\hrule
\vspace{10pt}
\begin{minipage}{230pt}
To prove $A \wedge B$, we have to prove both $A$ and $B$. \\

\vspace{5pt}

Therefore, we need to write (something like):

\vspace{20pt}
\end{minipage}
%
\hspace{15pt} \vline \hspace{15pt}
%
\begin{minipage}{300pt}
\begin{verbatim}
<1>1 A
    ...
<1>2 B
    ...
<1>q QED
    BY <1>1, <1>2
\end{verbatim}
\end{minipage}

%%%%%%%%%%%%%%%%%%%%%%%%%%%%%%%%%%%%%%%%%%%%%%%%%%%%%%%%%%%%%%%%%%%%%%%
\subsection{$\code{A \vee B}$}
\hrule
\vspace{10pt}
\begin{minipage}{230pt}
To prove $A \vee B$, it is enough to prove either $A$ or $B$. \\

\vspace{5pt}

Therefore, we need to write (something like):
\end{minipage}
%
\hspace{15pt} \vline \hspace{15pt}
%
\begin{minipage}{80pt}
\begin{verbatim}
<1>1 A
    ...
<1>q QED
    BY <1>1
\end{verbatim}
\end{minipage}
\hspace{5pt} \vline \hspace{25pt}
\begin{minipage}{80pt}
\begin{verbatim}
<1>1 B
    ...
<1>q QED
    BY <1>1
\end{verbatim}
\end{minipage}

%%%%%%%%%%%%%%%%%%%%%%%%%%%%%%%%%%%%%%%%%%%%%%%%%%%%%%%%%%%%%%%%%%%%%%
\subsection{$\code{A \Rightarrow B}$}
\hrule
\vspace{10pt}
\begin{minipage}{230pt}
To prove $A \Rightarrow B$, we have to prove $B$ under the assumption that $A$. \\

\vspace{5pt}

Therefore, we need to write (something like):
\vspace{10pt}
\end{minipage}
%
\hspace{15pt} \vline \hspace{15pt}
%
\begin{minipage}{80pt}
\begin{verbatim}
<1>1 ASSUME A
     PROVE  B
       ...
<1>q QED
    BY <1>1
\end{verbatim}
\vspace{1pt}
\end{minipage}
\hspace{5pt} \vline \hspace{15pt}
\begin{minipage}{80pt}
\begin{verbatim}
<1>1 SUFFICES
        ASSUME A
        PROVE  B
   ...
<1>q QED
    BY <1>1
\end{verbatim}
\end{minipage}


%%%%%%%%%%%%%%%%%%%%%%%%%%%%%%%%%%%%%%%%%%%%%%%%%%%%%%%%%%%%%%%%%%%%%%%
\subsection{$\code{\sim A}$}
\hrule
\vspace{10pt}
\begin{minipage}{230pt}
I think of proving $\sim A$, as proving $A \Rightarrow F$, where `$F$' 
is any contradiction; it could be `\verb|FALSE|', but it could also be 
something like $2 \leq 1$, or $a \le b \wedge b \le a$ for $a, b \in Int$ \\

\vspace{5pt}

Therefore, we need to write (something like):
\end{minipage}
%
\hspace{15pt} \vline \hspace{15pt}
%
\begin{minipage}{80pt}
\begin{verbatim}
<1>1 ASSUME A
     PROVE  F 
       ...
<1>q QED
    BY <1>1
\end{verbatim}
\vspace{5pt}
\end{minipage}



%%%%%%%%%%%%%%%%%%%%%%%%%%%%%%%%%%%%%%%%%%%%%%%%%%%%%%%%%%%%%%%%%%%%%%%
\subsection{$\code{\forall\ x : P(x)}$}
\hrule
\vspace{10pt}
\begin{minipage}{230pt}
Again, proving an universal quantifier is a lot like proving an
implication. To prove $\forall x : P(x)$ we need to prove
$P(x)$ for \emph{any} $x$, i.e. for an $x$ about which we
make no other assumptions.\\

\vspace{5pt}

The way to write that in TLA+ is (something like):
\end{minipage}
%
\hspace{15pt} \vline \hspace{15pt}
%
\begin{minipage}{80pt}
\begin{verbatim}
<1>1 ASSUME NEW x
     PROVE  P(x)
       ...
<1>q QED
    BY <1>1
\end{verbatim}
\end{minipage}


\subsubsection{$\code{\forall\ x \in S : P(x)}$}
\vspace{10pt}
\begin{minipage}{230pt}
When a universal quantifier is bounded, i.e. comes with a domain
assumption, that assumption can be used in the prove. So to prove
$\forall x \in S : P(x)$, we need to prove $P(x)$ for
\emph{any} $x \in S$, i.e. for an $x$ about which we make
no assumptions \emph{other than} that it is in $S$. \\


\vspace{5pt}

The way to write that in TLA+ is (something like):
\end{minipage}
%
\hspace{15pt} \vline \hspace{15pt}
%
\begin{minipage}{80pt}
\begin{verbatim}
<1>1 ASSUME NEW x \in S
     PROVE  P(x)
       ...
<1>q QED
    BY <1>1
\end{verbatim}
\end{minipage}


%%%%%%%%%%%%%%%%%%%%%%%%%%%%%%%%%%%%%%%%%%%%%%%%%%%%%%%%%%%%%%%%%%%%%%%
\subsection{$\code{\exists\ x : P(x)}$}
\hrule
\vspace{10pt}
\begin{minipage}{230pt}
To prove an existential quantifier we need to provide a witness, 
i.e. to prove $\exists x : P(x)$ we need to prove $P(a)$ for some
$a$ already in the context. \\

\vspace{5pt} 
In TLA+ we can do that, either by simply proving $P(a)$
for some $a$, or by first declaring which $a$ we will use, by means of
the \verb|WITNESS|-keyword, and then proving $P(a)$.
The way to write that in TLA+ is (something like):
\end{minipage}
%
\hspace{15pt} \vline \hspace{15pt}
%
\begin{minipage}{80pt}
\begin{verbatim}
<1>1 P(a)
    ...
<1>q QED
    BY <1>1
\end{verbatim}
\vspace{0pt}
\end{minipage}
\hspace{25pt} \vline \hspace{20pt}
\begin{minipage}{80pt}
\begin{verbatim}
<1> WITNESS a
<1>1 P(a)
    ...
<1>q QED
    BY <1>1
\end{verbatim}
\end{minipage}


\subsubsection{$\code{\exists\ x \in S : P(x)}$}
\vspace{10pt}
\begin{minipage}{230pt}
When an existential quantifier is bounded, i.e. comes with a domain
assumption, that assumption also needs to be proved. So to prove
$\exists x \in S : P(x)$, we need to prove both $a \in S$ and
$P(a)$ for some $a$ already in the context.\\

\vspace{5pt}
There are two things to notice about the \verb|WITNESS|-keyword here:
\begin{itemize}
  \item[a] Its syntax needs to correspond to the syntax of the
    existential quantifier it refers to; i.e. if the existential 
    is bounded, the \verb|WITNESS|-statement requires a bound.
    
  \item[b] A bounded \verb|WITNESS|-step discharges the domain
    assumption, i.e. the obligation $a \in S$. However, the step does
    not take a proof. So at this point, it needs to be obvious, that
    $a \in S$.  When a bounded \verb|WITNESS|-step does not turn
    green, it is a good idea to insert a step ``\verb|<1> a \in S|''
    and prove that step.
\end{itemize}

\end{minipage}
%
\hspace{15pt} \vline \hspace{15pt}
%
\begin{minipage}{80pt}
\begin{verbatim}
<1>1 a \in S
    ...
<1>2 P(a)
    ...
<1>q QED
    BY <1>1, <1>2
\end{verbatim}
%\vspace{35pt}
\end{minipage}
\hspace{25pt} \vline \hspace{15pt}
\begin{minipage}{80pt}
\begin{verbatim}
<1> WITNESS a \in S
<1>1 P(a)
    ...
<1>q QED
    BY <1>1
\end{verbatim}
\vspace{5pt}
\end{minipage}



%%%%%%%%%%%%%%%%%%%%%%%%%%%%%%%%%%%%%%%%%%%%%%%%%%%%%%%%%%%%%%%%%%%%%%%
%\subsection{$\Box A$}



%%%%%%%%%%%%%%%%%%%%%%%%%%%%%%%%%%%%%%%%%%%%%%%%%%%%%%%%%%%%%%%%%%%%%%%
%\subsection{$\Diamond A$}



%%%%%%%%%%%%%%%%%%%%%%%%%%%%%%%%%%%%%%%%%%%%%%%%%%%%%%%%%%%%%%%%%%%%%%%
%%%%%%%%%%%%%%            HOW DO I USE ?              %%%%%%%%%%%%%%%%%
%%%%%%%%%%%%%%%%%%%%%%%%%%%%%%%%%%%%%%%%%%%%%%%%%%%%%%%%%%%%%%%%%%%%%%%
\newpage
\section{How Do I Use ... ?}


%%%%%%%%%%%%%%%%%%%%%%%%%%%%%%%%%%%%%%%%%%%%%%%%%%%%%%%%%%%%%%%%%%%%%%%
\subsection{$\code{A \wedge B}$}
\hrule
\vspace{10pt}
\begin{minipage}{230pt}
To use $A \wedge B$, we don't have to actually do anything;
$A \wedge B$ in the context is as good as each $A$ and $B$ in
the context independently. \\

\vspace{5pt}

$A \wedge B$ enables us to write both:
\end{minipage}
%
\hspace{15pt} \vline \hspace{15pt}
%
\begin{minipage}{80pt}
\begin{verbatim}
<1>1 A
    OBVIOUS
\end{verbatim}
\end{minipage}
\hspace{5pt} \vline \hspace{15pt}
\begin{minipage}{80pt}
\begin{verbatim}
<1>1 B
    OBVIOUS
\end{verbatim}
\end{minipage}


%%%%%%%%%%%%%%%%%%%%%%%%%%%%%%%%%%%%%%%%%%%%%%%%%%%%%%%%%%%%%%%%%%%%%%%
\subsection{$\code{A \vee B}$}
\hrule
\vspace{10pt}
\begin{minipage}{230pt}
To use $A \vee B$, we have to prove the current goal twice, 
once assuming $A$, and once assuming $B$. \\

\vspace{5pt}

Probably the simplest way to use $A \vee B$, is by means of the
\verb|CASE|-keyword:

\vspace{15pt}
\end{minipage}
%
\hspace{15pt} \vline \hspace{15pt}
%
\begin{minipage}{80pt}
\begin{verbatim}
<1>1 CASE A
    ...
<1>2 CASE B
    ...
<1>q QED
    BY <1>1, <1>2
\end{verbatim}
\end{minipage}


%%%%%%%%%%%%%%%%%%%%%%%%%%%%%%%%%%%%%%%%%%%%%%%%%%%%%%%%%%%%%%%%%%%%%%%
\subsection{$\code{A \Rightarrow B}$}
\hrule
\vspace{10pt}
\begin{minipage}{230pt}
$A \Rightarrow B$ in the context means that, as soon as we have proved
$A$, we can get $B$ straight away.  \\

\vspace{5pt}
$A \Rightarrow B$ in the context enables us to write:

\vspace{15pt}
\end{minipage}
%
\hspace{15pt} \vline \hspace{15pt}
%
\begin{minipage}{80pt}
\begin{verbatim}
<1>1 A
    ...
<1>2 B
    BY <1>1
\end{verbatim}
\end{minipage}


%%%%%%%%%%%%%%%%%%%%%%%%%%%%%%%%%%%%%%%%%%%%%%%%%%%%%%%%%%%%%%%%%%%%%%%
\subsection{$\code{\sim A}$}
\hrule
\vspace{10pt}
\begin{minipage}{230pt}
Again, I think it is useful to think of $\sim A$ as $A \Rightarrow F$,
for any $F$. That means that, if we have $\sim A$ in the context, and 
we manage to prove $A$, we are done. \\

\vspace{5pt}

$\sim A$ in the context enables us to write:
\end{minipage}
%
\hspace{15pt} \vline \hspace{15pt}
%
\begin{minipage}{80pt}
\begin{verbatim}
<1>1 A
     ...
<1>q QED
    BY <1>1
\end{verbatim}
\vspace{5pt}
\end{minipage}


%%%%%%%%%%%%%%%%%%%%%%%%%%%%%%%%%%%%%%%%%%%%%%%%%%%%%%%%%%%%%%%%%%%%%%%
\subsection{$\code{\forall\ x : P(x)}$}
\hrule
\vspace{10pt}
\begin{minipage}{230pt}
$\forall x : P(x)$ in the context allows us to assert $P(a)$ for 
any $a$ in the context.
\vspace{5pt}

That is to say, for all \verb|a| in the context, it enables us to write:
\end{minipage}
%
\hspace{15pt} \vline \hspace{15pt}
%
\begin{minipage}{80pt}
\begin{verbatim}
<1>1 P(a)
  OBVIOUS
\end{verbatim}
\vspace{5pt}
\end{minipage}


\subsubsection{$\code{\forall \ x \in S : P(x)}$}
\vspace{10pt}
\begin{minipage}{230pt}
A bounded universal quantifier of the form $\forall x \in S : P(x)$,
in the context allows us to assert $P(a)$ for 
any $a$, as long as we can prove that $a \in S$.
\vspace{5pt}

That is to say, for all \verb|a| in the context, it enables us to write:
\end{minipage}
%
\hspace{15pt} \vline \hspace{15pt}
%
\begin{minipage}{80pt}
\begin{verbatim}
<1>1 a \in S
    ...
<1>2 P(a)
  BY <1>1
\end{verbatim}
\vspace{5pt}
\end{minipage}


%%%%%%%%%%%%%%%%%%%%%%%%%%%%%%%%%%%%%%%%%%%%%%%%%%%%%%%%%%%%%%%%%%%%%%%
\newpage
\subsection{$\code{\exists\ x : P(x)}$}
\hrule
\vspace{10pt}
\begin{minipage}{230pt}
An existential quantifier like $\exists x : P(x)$ in the context
allows us to introduce a new $a$ into the context, and assert that
$P(a)$.
\vspace{5pt}

There are different ways of doing that. One is with a simple
\verb|ASSUME/PROVE|; in that case the current goal \verb|G| has to be
repeated.  An alternative is to use the \verb|PICK|-keyword, which in
this situation is short-hand for \verb|SUFFICES-ASSUME/PROVE|.

\end{minipage}
%
\hspace{15pt} \vline \hspace{15pt}
%
%
\begin{minipage}{100pt}
\begin{verbatim}
<1>1 ASSUME NEW a,
            P(a)
     PROVE  G
      ...
<1>q QED
    BY <1>1
\end{verbatim}
\end{minipage}
\hspace{0pt} \vline \hspace{10pt}
\begin{minipage}{80pt}
\begin{verbatim}
<1>1 PICK a : P(a)
    OBVIOUS




\end{verbatim}
\end{minipage}

\subsection{$\code{\exists\ x \in S : P(x)}$}
\vspace{10pt}
\begin{minipage}{230pt}
A bounded existential quantifier $\exists x \in S : P(x)$ in the context
allows us to introduce a new $a$ into the context, and assert both 
that $a \in S$ and that $P(a)$.

\vspace{5pt}

The additional fact that $a \in S$ is added to the
\verb|ASSUME/PROVE|-step, as you would expect. 
The \verb|PICK|-keyword also looks as we would expect. 
\end{minipage}
%
\hspace{15pt} \vline \hspace{5pt}
%
%
\begin{minipage}{100pt}
\begin{verbatim}
<1>1 ASSUME NEW a,
            a \in S,
            P(a)
     PROVE  G
      ...
<1>q QED
    BY <1>1
\end{verbatim}
\end{minipage}
\hspace{10pt} \vline \hspace{10pt}
\begin{minipage}{80pt}
\begin{verbatim}
<1>1 PICK a \in S : P(a)
    OBVIOUS





\end{verbatim}
\end{minipage}




%%%%%%%%%%%%%%%%%%%%%%%%%%%%%%%%%%%%%%%%%%%%%%%%%%%%%%%%%%%%%%%%%%%%%%%
%\subsection{$\Box A$}



%%%%%%%%%%%%%%%%%%%%%%%%%%%%%%%%%%%%%%%%%%%%%%%%%%%%%%%%%%%%%%%%%%%%%%%
%\subsection{$\Diamond A$}



%%%%%%%%%%%%%%%%%%%%%%%%%%%%%%%%%%%%%%%%%%%%%%%%%%%%%%%%%%%%%%%%%%%%%%%
%%%%%%%%%%%%%%%         FREQUENT SITUATIONS           %%%%%%%%%%%%%%%%%
%%%%%%%%%%%%%%%%%%%%%%%%%%%%%%%%%%%%%%%%%%%%%%%%%%%%%%%%%%%%%%%%%%%%%%%
\newpage
\section{Why Doesn't This Turn Green?}



\end{document}
